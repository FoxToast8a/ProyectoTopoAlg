%!TEX root = ../main.tex
Luego de nuestro paso por la idea geométrica y teniendo en cuenta que definimos el grupo de trenzas geométricas, en esta sección abordaremos las trenzas desde un punto de vista netamente algebraico.
\begin{definition}
    El grupo de trenzas de Artin $B_n$ es el grupo generado por $n-1$ generadores $\sigma_i$ con $i=1,2,\ldots n-1$ y las relaciones
    $$\sigma_i\sigma_j=\sigma_j\sigma_i,$$
    para todo $i,j=1,2,\ldots n-1$ con $|i-j|\geq 2,$ y
    $$\sigma_i\sigma_{i+1}\sigma_i=\sigma_{i+1}\sigma_i\sigma_{i+1},$$
    para $i=1,2,\ldots n-2.$
\end{definition}
Dicho esto veamos los ejemplos mas sencillos de grupo de trenzas.
\begin{eg}
 $B_1=\{1\}$, es decir el grupo trivial ya que no habrían generadores, en el caso de $B_2$ tenemos que es un grupo con un generador $\sigma_1$, donde nuevamente no hay relaciones, así $B_2=\langle\sigma_1\rangle\cong \Z$.
\end{eg}
Como es comun en grupos queremos ver que como se comportan sus elementos bajo homomorfismos. Así sea $f:B_n\to G$ un homomorfismo de grupos, por ser homomorfismo, es claro que $f(\{\sigma_1,\ldots,\sigma_{n-1}\})=\{s_1,\ldots,s_{n-1}\}$ debe cumplir las relaciones de trenzas, pero también tenemos la dirección contraria
\begin{lemma}
    Si existen elementos $\{s_i|i=1,\ldots,n-1\}$ en un grupo $G$ tales que se satisfacen las relaciones de trenzas, entonces existe un unico homomorfismo de grupos $f:B_n\to G$, tal que $s_i=f(\sigma_i).$
\end{lemma}
\begin{proof}
    \textcolor{blue}{luego}
\end{proof}
Para los primeros dos grupos de trenzas, estos eran abelianos, pero esto de hecho no se mantiene siempre para esto mostremos primero el siguiente lema
\begin{lemma}
    Las trasposiciones simples $s_i=(i\,i+1)\in S_n$, generan $S_n$ y satisfacen las relaciones de trenzas.
\end{lemma}
\begin{proof}
    \textcolor{blue}{luego}
\end{proof}
\begin{theorem}
    El grupo $B_n$ para $n\geq 3$ no es abeliano
\end{theorem}
\begin{proof}
    \textcolor{blue}{luego}
\end{proof}
Por lo visto antes existe un homomorfismo unico de grupos $\pi:B_n\to S_n$, de manera similar es claro que la inclusion $i:B_n\to B_{n+1}$ es un homomorfismo, esto en ocasiones resulta util ya que tendremos una cadena de subgrupos $B_1\subset B_2\subset B_3\subset \dots$ y un diagrama que conmuta
\begin{center}
    

\tikzset{every picture/.style={line width=0.75pt}} %set default line width to 0.75pt        

\begin{tikzpicture}[x=0.75pt,y=0.75pt,yscale=-1.5,xscale=1.5]
%uncomment if require: \path (0,300); %set diagram left start at 0, and has height of 300

%Straight Lines [id:da38871465122456583] 
\draw    (150,80) -- (218,80) ;
\draw [shift={(220,80)}, rotate = 180] [color={rgb, 255:red, 0; green, 0; blue, 0 }  ][line width=0.75]    (4.37,-1.32) .. controls (2.78,-0.56) and (1.32,-0.12) .. (0,0) .. controls (1.32,0.12) and (2.78,0.56) .. (4.37,1.32)   ;
%Straight Lines [id:da5491987648817868] 
\draw    (150,150) -- (218,150) ;
\draw [shift={(220,150)}, rotate = 180] [color={rgb, 255:red, 0; green, 0; blue, 0 }  ][line width=0.75]    (4.37,-1.32) .. controls (2.78,-0.56) and (1.32,-0.12) .. (0,0) .. controls (1.32,0.12) and (2.78,0.56) .. (4.37,1.32)   ;
%Straight Lines [id:da16177838772087816] 
\draw    (130,90) -- (130,138) ;
\draw [shift={(130,140)}, rotate = 270] [color={rgb, 255:red, 0; green, 0; blue, 0 }  ][line width=0.75]    (4.37,-1.32) .. controls (2.78,-0.56) and (1.32,-0.12) .. (0,0) .. controls (1.32,0.12) and (2.78,0.56) .. (4.37,1.32)   ;
%Straight Lines [id:da023913099195564946] 
\draw    (240,90) -- (240,138) ;
\draw [shift={(240,140)}, rotate = 270] [color={rgb, 255:red, 0; green, 0; blue, 0 }  ][line width=0.75]    (4.37,-1.32) .. controls (2.78,-0.56) and (1.32,-0.12) .. (0,0) .. controls (1.32,0.12) and (2.78,0.56) .. (4.37,1.32)   ;

% Text Node
\draw (121,70.4) node [anchor=north west][inner sep=0.75pt]    {$B_{n}$};
% Text Node
\draw (229,70.4) node [anchor=north west][inner sep=0.75pt]    {$S_{n}$};
% Text Node
\draw (115,140.4) node [anchor=north west][inner sep=0.75pt]    {$B_{n+1}$};
% Text Node
\draw (229,140.4) node [anchor=north west][inner sep=0.75pt]    {$S_{n+1}$};
% Text Node
\draw (93,102.4) node [anchor=north west][inner sep=0.75pt]    {$\dotsc $};
% Text Node
\draw (253,102.4) node [anchor=north west][inner sep=0.75pt]    {$\dotsc $};


\end{tikzpicture}
\end{center}
