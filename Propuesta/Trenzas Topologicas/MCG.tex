%!TEX root = ../main.tex

Sea $M$ una variedad topologica orientada y $Q\subset\text{Int}(M)$ finito.

\begin{definition}
    Un \textit{auto-homeomorfismo} de la pareja $(M,Q)$ es un homeomorfismo $f:M\to M$ tal que
    \begin{itemize}
        \item[i)] Para todo $x\in\partial M$,\, $f(x)=x$.
        \item[ii)] $f(Q)=Q.$ 
        \item[iii)] Preserva la orientacion.
    \end{itemize}
\end{definition}

Observe que en $i)$ cada punto de frontera esta fijo, mientras que por $ii)$ puede que $Q$ quede fijo pero tambien esta la posibilidad de permutar los puntos en $Q.$ 

Como es constumbre tenemos que definir una nocion de isotopia entre estos auto-homemomorfismos para poder empezar a dar la estructura adecuada y no estar repitiendo elementos.

\begin{definition}
    Sean $f_0,f_1$ auto-homeomorfismos de $(M,Q)$, decimos que son isotopicos si existe una familia de auto-homemomorfismos $\{f_t\}_{t\in I}$ donde estos sean los extremos y cada $f_t$ sea continuo.
\end{definition}
Dada la similitud con nociones anteriores no mostraremos que efectivamente esta isotopia define una relacion de equivalencia.
\begin{definition}
    El \textit{Mapping Class Group} $\mathcal{M}(M,Q)$ es el conjunto de clases de isotopia de automorfismos con la composicion de funciuones como operacion.
\end{definition}
Si el conjunto de puntos $Q=\varnothing$ omitiremos su escritura en general. A continuacion veremos un ejemplo clasico de este grupo
\begin{eg}
    Mostraremos que $\mathcal{M}(D^n)=\{1\},$ para esto consideremos la siguiente familia de funciones
    $$h_t(x)=\begin{cases}
        x&\text{si }t\leq|x|\leq 1,\\
        th\left(\dfrac{x}{t}\right)&\text{si }|x|<t.
    \end{cases}$$
   Donde $h$ es un auto-homemomorfismo de $D^n$. Note que $h_0=Id_{D^n}$, mientras que
   $$h_1=\begin{cases}
       x&\text{si }|x|=1,\\
       h(x)&\text{si }|x|<1.
   \end{cases}$$ 
   Que justamente es $h$ ya que este es auto-homeomorfismo. Claramente estos son continuos y son auto-homeomorfismos, asi concluimos lo deseado, en particular note que si $h(0)=0$, cada $h_t$ fija $0$ por lo que $\mathcal{M}(D^n,\{0\})=\{1\}.$
\end{eg}
\textcolor{blue}{hablar de halftwist}