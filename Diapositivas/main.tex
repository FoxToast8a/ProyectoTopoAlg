\documentclass[xcolor=dvipsnames,10pt]{beamer}
\usepackage[utf8]{inputenc}
\usepackage[spanish]{babel}
\usepackage{multirow,rotating}
\usepackage{amsfonts,amsmath,amssymb,amsthm}
\usepackage{color}
\usepackage{hyperref}
\usepackage{fourier,libertine}
\usepackage{transparent}
\usepackage{tikz-cd}
\usepackage{array}
\usepackage{algpseudocode}
\usepackage{siunitx}
\usepackage{algorithm}
\usepackage{mathtools,nccmath}%
\usetheme{Madrid}
\usepackage{xcolor}
\usepackage{natbib}
\usepackage{hyperref}
\usepackage{ragged2e}
\usefonttheme{serif}
\usefonttheme{professionalfonts}
\newcommand\T{\ensuremath{\mathbb{T}}}
\newcommand\N{\ensuremath{\mathbb{N}}}
\newcommand\R{\ensuremath{\mathbb{R}}}
\newcommand\Z{\ensuremath{\mathbb{Z}}}
\renewcommand\O{\ensuremath{\emptyset}}
\newcommand\Q{\ensuremath{\mathbb{Q}}}
\newcommand\C{\ensuremath{\mathbb{C}}}
\newcommand\Hs{\ensuremath{\mathbb{H}}}
% set color ------------------------------------------------------------------
\definecolor{DarkBlue}{rgb}{0.04706, 0.13725, 0.26667} 
\definecolor{cadmiumred}{rgb}{0.45 , 0.12, 0.23}
\definecolor{armygreen}{rgb}{0.10, 0.27, 0.19}
\definecolor{coolblack}{rgb}{0.0, 0.18, 0.39}
\definecolor{lilac}{rgb}{0.33, 0.12, 0.36}
\definecolor{negro}{rgb}{0, 0, 0}
\usecolortheme[named=coolblack]{structure}
\setbeamercolor{block title}{bg=coolblack!100!white,fg=white}
\setbeamercolor{block body}{bg=coolblack!11!white}

%----------------------------------------------------------------------------
\setbeamerfont{title}{size=\large}
\setbeamerfont{subtitle}{size=\small}
\setbeamerfont{author}{size=\small}
\setbeamerfont{date}{size=\footnotesize}
\setbeamerfont{institute}{size=\footnotesize}
\title[Universidad Nacional de Colombia]{Grupos de Trenzas y Espacios de Configuración}
\subtitle{Un primer acercamiento a las nociones algebraicas, geométricas y topológicas}
\author[Topología Algebraica]{
Edgar Santiago Ochoa Quiroga}

\date[\textcolor{white}{Diciembre/2025}]


\begin{document}
\maketitle
\begin{frame}{Introducción}
Cuando hablamos de Trenzas en un ámbito general es completamente natural que la primera imagen que se pase por la cabeza es la de una trenza de cabello. La mas clásica se hace agarrando tres mechones, donde los mechones externos pasan hacia el centro intercalándose.  
\begin{figure}
    \centering
    \includegraphics[width=0.5\linewidth]{trenzado_tres.jpg}
    
\end{figure}
La primera aparición con mención propia de los grupos de trenzas se las debemos a Emil Artin, quien en 1925 los introdujo para modelar como se entrelazaban múltiples cuerdas en un espacio euclidiano 3-dimensional, estas cuerdas es lo que conocemos como trenzas.
\end{frame}

\begin{frame}{Grupos de Trenzas geometricos $\mathcal{B}_n$}
\begin{block}{Definición}
    Una \textbf{trenza geometrica} con $n\geq 1$ cuerdas es un conjunto $b\subset \R^2\times I$ formado por $n$ intervalos topológicos disyuntos llamados las \textit{cuerdas} de $b$, tales que la proyeccion $\R^2\times I\to I$, envia cada cuerda de manera homeomorfica a $I$, y ademas
     \begin{align*}
         b\cap(\R^2\times \{0\})=\{(i,0,0)|i=1,2,\ldots n\},\\
         b\cap(\R^2\times \{0\})=\{(i,0,1)|i=1,2,\ldots n\}.
     \end{align*}
\end{block}
\end{frame}
\begin{frame}
      \begin{center}
          %!TEX root = ./main.tex

\tikzset{every picture/.style={line width=0.75pt}} %set default line width to 0.75pt        

\begin{tikzpicture}[x=0.75pt,y=0.75pt,yscale=-2,xscale=2]
%uncomment if require: \path (0,300); %set diagram left start at 0, and has height of 300

%Straight Lines [id:da15571602795499961] 
\draw    (230,70) -- (230,150) ;
%Straight Lines [id:da6387164240227489] 
\draw    (250.5,70.5) -- (250,150) ;
%Straight Lines [id:da17127845146372145] 
\draw    (270,70) -- (270,150) ;
%Straight Lines [id:da1879833889944229] 
\draw    (291,70) -- (290,150) ;
%Straight Lines [id:da29818824665085064] 
\draw [color={rgb, 255:red, 155; green, 155; blue, 155 }  ,draw opacity=1 ]   (209.5,69) -- (209.99,169) ;
\draw [shift={(210,171)}, rotate = 269.72] [color={rgb, 255:red, 155; green, 155; blue, 155 }  ,draw opacity=1 ][line width=0.75]    (10.93,-3.29) .. controls (6.95,-1.4) and (3.31,-0.3) .. (0,0) .. controls (3.31,0.3) and (6.95,1.4) .. (10.93,3.29)   ;
%Straight Lines [id:da9999252093097876] 
\draw [color={rgb, 255:red, 155; green, 155; blue, 155 }  ,draw opacity=1 ]   (209.5,69) -- (318,69.98) ;
\draw [shift={(320,70)}, rotate = 180.52] [color={rgb, 255:red, 155; green, 155; blue, 155 }  ,draw opacity=1 ][line width=0.75]    (10.93,-3.29) .. controls (6.95,-1.4) and (3.31,-0.3) .. (0,0) .. controls (3.31,0.3) and (6.95,1.4) .. (10.93,3.29)   ;
%Straight Lines [id:da5087320786071492] 
\draw [color={rgb, 255:red, 155; green, 155; blue, 155 }  ,draw opacity=1 ]   (210,70) -- (268.21,40.89) ;
\draw [shift={(270,40)}, rotate = 153.43] [color={rgb, 255:red, 155; green, 155; blue, 155 }  ,draw opacity=1 ][line width=0.75]    (10.93,-3.29) .. controls (6.95,-1.4) and (3.31,-0.3) .. (0,0) .. controls (3.31,0.3) and (6.95,1.4) .. (10.93,3.29)   ;
%Straight Lines [id:da6250989615981395] 
\draw [color={rgb, 255:red, 155; green, 155; blue, 155 }  ,draw opacity=1 ]   (210,70) -- (190,80) ;
%Straight Lines [id:da8602700993161204] 
\draw [color={rgb, 255:red, 155; green, 155; blue, 155 }  ,draw opacity=1 ]   (209.5,69) -- (189.5,69) ;
%Straight Lines [id:da31711673101036175] 
\draw [color={rgb, 255:red, 155; green, 155; blue, 155 }  ,draw opacity=1 ]   (320,150) -- (190,150) ;

% Text Node
\draw (159,63.4) node [anchor=north west][inner sep=0.75pt]  [font=\scriptsize]  {$t=0$};
% Text Node
\draw (162,143.4) node [anchor=north west][inner sep=0.75pt]  [font=\scriptsize]  {$t=1$};
% Text Node
\draw (197,158.4) node [anchor=north west][inner sep=0.75pt]  [font=\scriptsize]  {$t$};
% Text Node
\draw (318,54.4) node [anchor=north west][inner sep=0.75pt]  [font=\scriptsize]  {$x$};
% Text Node
\draw (276,33.4) node [anchor=north west][inner sep=0.75pt]  [font=\scriptsize]  {$y$};


\end{tikzpicture}
      \end{center}
  \end{frame}
\begin{frame}
     \begin{center}
          %!TEX root = ./main.tex

\tikzset{every picture/.style={line width=0.75pt}} %set default line width to 0.75pt        

\begin{tikzpicture}[x=0.75pt,y=0.75pt,yscale=-1.5,xscale=1.5]
%uncomment if require: \path (0,300); %set diagram left start at 0, and has height of 300

%Straight Lines [id:da6387164240227489] 
\draw    (250.5,70.5) -- (250,110) ;
%Straight Lines [id:da1879833889944229] 
\draw    (291,70) -- (290,90) ;
%Straight Lines [id:da29818824665085064] 
\draw [color={rgb, 255:red, 155; green, 155; blue, 155 }  ,draw opacity=1 ]   (209.5,69) -- (209.99,169) ;
\draw [shift={(210,171)}, rotate = 269.72] [color={rgb, 255:red, 155; green, 155; blue, 155 }  ,draw opacity=1 ][line width=0.75]    (10.93,-3.29) .. controls (6.95,-1.4) and (3.31,-0.3) .. (0,0) .. controls (3.31,0.3) and (6.95,1.4) .. (10.93,3.29)   ;
%Straight Lines [id:da9999252093097876] 
\draw [color={rgb, 255:red, 155; green, 155; blue, 155 }  ,draw opacity=1 ]   (209.5,69) -- (318,69.98) ;
\draw [shift={(320,70)}, rotate = 180.52] [color={rgb, 255:red, 155; green, 155; blue, 155 }  ,draw opacity=1 ][line width=0.75]    (10.93,-3.29) .. controls (6.95,-1.4) and (3.31,-0.3) .. (0,0) .. controls (3.31,0.3) and (6.95,1.4) .. (10.93,3.29)   ;
%Straight Lines [id:da5087320786071492] 
\draw [color={rgb, 255:red, 155; green, 155; blue, 155 }  ,draw opacity=1 ]   (210,70) -- (268.21,40.89) ;
\draw [shift={(270,40)}, rotate = 153.43] [color={rgb, 255:red, 155; green, 155; blue, 155 }  ,draw opacity=1 ][line width=0.75]    (10.93,-3.29) .. controls (6.95,-1.4) and (3.31,-0.3) .. (0,0) .. controls (3.31,0.3) and (6.95,1.4) .. (10.93,3.29)   ;
%Straight Lines [id:da6250989615981395] 
\draw [color={rgb, 255:red, 155; green, 155; blue, 155 }  ,draw opacity=1 ]   (210,70) -- (190,80) ;
%Straight Lines [id:da8602700993161204] 
\draw [color={rgb, 255:red, 155; green, 155; blue, 155 }  ,draw opacity=1 ]   (209.5,69) -- (189.5,69) ;
%Straight Lines [id:da31711673101036175] 
\draw [color={rgb, 255:red, 155; green, 155; blue, 155 }  ,draw opacity=1 ]   (320,150) -- (190,150) ;
%Straight Lines [id:da7048848459711928] 
\draw    (250.5,122.5) -- (250,150) ;
%Curve Lines [id:da36065808527822485] 
\draw    (230,70) .. controls (228.5,80.5) and (231,88) .. (246,89) ;
%Curve Lines [id:da5874880850528349] 
\draw    (254,90) .. controls (285.5,123) and (237,98) .. (230,150) ;
%Curve Lines [id:da3625971369635478] 
\draw    (270,70) .. controls (279,103) and (360,115) .. (298,130) ;
%Straight Lines [id:da02942943718787716] 
\draw    (291,100) -- (290,150) ;
%Curve Lines [id:da5037121904515169] 
\draw    (270,150) .. controls (270.5,130.5) and (273.5,135) .. (283,131) ;

% Text Node
\draw (159,63.4) node [anchor=north west][inner sep=0.75pt]  [font=\scriptsize]  {$t=0$};
% Text Node
\draw (162,143.4) node [anchor=north west][inner sep=0.75pt]  [font=\scriptsize]  {$t=1$};
% Text Node
\draw (197,158.4) node [anchor=north west][inner sep=0.75pt]  [font=\scriptsize]  {$t$};
% Text Node
\draw (318,54.4) node [anchor=north west][inner sep=0.75pt]  [font=\scriptsize]  {$x$};
% Text Node
\draw (276,33.4) node [anchor=north west][inner sep=0.75pt]  [font=\scriptsize]  {$y$};


\end{tikzpicture}
      \end{center}
\end{frame}
\begin{frame}{Equivalencia de Trenzas Geometricas}
    \begin{block}{Definición}
        Dos trenzas $b$ y $b^\prime$ son isotopicas si existe una función continua $F:b\times I\to R^2\times I$, tal que para cada $s\in I$, la función $F_s:=F(-,s)$ es un embedimiento el cual su imagen es una trenza geométrica de $n$ cuerdas, $F_0=Id_b$ y $F_1(b)=b^\prime.$\\
    Tanto $F$ como la familia de trenzas geometricas $\{F_s(b)\}_{s\in I}$ se conocen como una isotopia de $b$ a $b^\prime$.
    \end{block}
    \begin{block}{Proposición}
        La isotopia entre trenzas geométricas define una relación de equivalencia.
    \end{block}
\end{frame}
\begin{frame}{Estructura de Monoide}

\begin{block}{Definicion}
    Dadas 2 trenzas geométricas con $n$ cuerdas $b_1,b_2$, definimos su producto como el conjunto de puntos $(x,y,t)\in \R^2\times I$ tales que si $t\in[0,1/2]$ entonces $(x,y,2t)\in b_1$, y en caso donde $t\in[1/2,1]$ tenemos que $(x,y,2t-1)\in b_2$.
\end{block}
\begin{block}{Proposición}
    Dadas $b_1,b_2,b_1^\prime,b_2^\prime$, donde $b_i$ es isotopica a $b_i^\prime$ para $i=1,2$, entonces $b_1b_2$ es isotopica a $b_1^\prime b^\prime_2$. 
\end{block}
\end{frame}
\begin{frame}{Diagramas de Trenzas}
\begin{block}{Definición}
    Un \textbf{diagrama de trenzas} de $n$ cuerdas es un conjunto $D\subset \R\times I$, vista como la union de $n$ intervalos topologicos llamados las cuerdas de $D$ donde se cumplen las siguientes condiciones.
    \begin{itemize}
        \item La proyeccion $\R\times I\to I$ envia cada cuerda homeomorficamente a $I.$
        \item Cada punto $\{1,2,\ldots\}\times\{0,1\}$ es el extremo de una unica cuerda.
        \item Todo punto de $\R\times I$ pertenece a lo maximo a 2 cuerdas. En cada punto de interseccion las cuerdas se cruzan de manera transversal y una de ellas se difumina para indicar que pasa por debajo, mientras la que se ve continua indica que pasa por arriba. 
        \end{itemize} 
\end{block}
\end{frame}
\begin{frame}
    \begin{center}
        %!TEX root = ./main.tex


\tikzset{every picture/.style={line width=0.75pt}} %set default line width to 0.75pt        

\begin{tikzpicture}[x=0.75pt,y=0.75pt,yscale=-1.5,xscale=1.5]
%uncomment if require: \path (0,300); %set diagram left start at 0, and has height of 300

%Straight Lines [id:da9999252093097876] 
\draw [color={rgb, 255:red, 155; green, 155; blue, 155 }  ,draw opacity=1 ]   (209.5,69) -- (318,69.98) ;
\draw [shift={(320,70)}, rotate = 180.52] [color={rgb, 255:red, 155; green, 155; blue, 155 }  ,draw opacity=1 ][line width=0.75]    (10.93,-3.29) .. controls (6.95,-1.4) and (3.31,-0.3) .. (0,0) .. controls (3.31,0.3) and (6.95,1.4) .. (10.93,3.29)   ;
%Straight Lines [id:da8602700993161204] 
\draw [color={rgb, 255:red, 155; green, 155; blue, 155 }  ,draw opacity=1 ]   (209.5,69) -- (189.5,69) ;
%Straight Lines [id:da31711673101036175] 
\draw [color={rgb, 255:red, 155; green, 155; blue, 155 }  ,draw opacity=1 ]   (320,150) -- (190,150) ;
%Straight Lines [id:da02070104769775316] 
\draw    (250,70) -- (250,150) ;
%Curve Lines [id:da05791226204824307] 
\draw    (221,70) .. controls (222.4,75.4) and (227.2,91) .. (245,100) ;
%Curve Lines [id:da40088925285683386] 
\draw    (280,70) .. controls (273,89.8) and (268,95.4) .. (256,100) ;
%Curve Lines [id:da876049615716019] 
\draw    (219,150) .. controls (221.4,126.2) and (223.8,111) .. (245,104) ;
%Curve Lines [id:da5837647469150872] 
\draw    (256,105) .. controls (268.4,108.2) and (279,121.4) .. (280,150) ;

% Text Node
\draw (159,63.4) node [anchor=north west][inner sep=0.75pt]  [font=\scriptsize]  {$t=0$};
% Text Node
\draw (162,143.4) node [anchor=north west][inner sep=0.75pt]  [font=\scriptsize]  {$t=1$};
% Text Node
\draw (197,158.4) node [anchor=north west][inner sep=0.75pt]  [font=\scriptsize]  {$t$};
% Text Node
\draw (318,54.4) node [anchor=north west][inner sep=0.75pt]  [font=\scriptsize]  {$x$};


\end{tikzpicture}
    \end{center}
\end{frame}
\begin{frame}{}
    \begin{block}{Definicion}
        Dos diagramas de trenzas $D$ y $D^\prime$ se dicen isotopicos  si existe una funcion continua $F:D\times I\to R\times I$ tal que para cada $s\in I$, $D_s=F(D\times\{s\})\subset \R\times I$ es un diagrama de trenzas con la misma cantidad cuerdas, ademas $D_0=D$ y $D_1=D^\prime.$ 
    \end{block}

    \begin{block}{Definición}
        Definimos $\Omega_2$ un movimiento donde tomamos dos cuerdas del diagrama y creamos dos cruces nuevos transversales, pasando una de las cuerdas del diagrama por debajo de la otra, mientras que a $\Omega_3$ es un movimiento que involucra a tres cuerdas y preserva el numero de cruces transversales pero invierte el diagrama de manera reflexiva.  
    \end{block}
\end{frame}
\begin{frame}
    \begin{center}
        
%!TEX root = ./main.tex

\tikzset{every picture/.style={line width=0.75pt}} %set default line width to 0.75pt        

\begin{tikzpicture}[x=0.75pt,y=0.75pt,yscale=-1.5,xscale=1.5]
%uncomment if require: \path (0,300); %set diagram left start at 0, and has height of 300

%Curve Lines [id:da467535346440173] 
\draw    (100,100) .. controls (101.8,108.2) and (106.2,115.8) .. (119,121) ;
%Curve Lines [id:da18360277031656336] 
\draw    (140,100) .. controls (139.8,126.2) and (100.2,121.8) .. (100,180) ;
%Curve Lines [id:da8935762898756716] 
\draw    (127,127) .. controls (138.6,145.4) and (141.4,156) .. (140,180) ;
%Curve Lines [id:da11204595747175461] 
\draw    (159.67,100) .. controls (161.47,108.2) and (161.2,139.8) .. (174,145) ;
%Curve Lines [id:da9612794099188988] 
\draw    (199.67,100) .. controls (199.8,150.6) and (160.53,153.2) .. (160,180) ;
%Curve Lines [id:da9839227280377414] 
\draw    (186,150) .. controls (197.4,155.8) and (199.73,172) .. (200,180) ;
%Straight Lines [id:da021068853778946184] 
\draw [color={rgb, 255:red, 155; green, 155; blue, 155 }  ,draw opacity=1 ]   (220,140) -- (248,140) ;
\draw [shift={(250,140)}, rotate = 180] [color={rgb, 255:red, 155; green, 155; blue, 155 }  ,draw opacity=1 ][line width=0.75]    (6.56,-1.97) .. controls (4.17,-0.84) and (1.99,-0.18) .. (0,0) .. controls (1.99,0.18) and (4.17,0.84) .. (6.56,1.97)   ;
%Straight Lines [id:da95151298393968] 
\draw [color={rgb, 255:red, 155; green, 155; blue, 155 }  ,draw opacity=1 ]   (220,136) -- (220,144) ;
%Curve Lines [id:da5609149547400984] 
\draw    (260,100) .. controls (261.8,108.2) and (264.2,129.8) .. (277,135) ;
%Curve Lines [id:da2754758327066825] 
\draw    (300,100) .. controls (299.8,139.8) and (260.6,139) .. (260,180) ;
%Curve Lines [id:da8793140836193352] 
\draw    (289,140) .. controls (299.4,150.2) and (299.4,162.2) .. (300,180) ;
%Curve Lines [id:da24663066435532877] 
\draw    (320,100) .. controls (321.8,108.2) and (324.2,129.8) .. (337,135) ;
%Curve Lines [id:da0768559780876632] 
\draw    (360,100) .. controls (359.8,139.8) and (320.6,139) .. (320,180) ;
%Curve Lines [id:da2528200653455043] 
\draw    (349,140) .. controls (359.4,150.2) and (359.4,162.2) .. (360,180) ;
%Straight Lines [id:da4894361551183606] 
\draw [color={rgb, 255:red, 155; green, 155; blue, 155 }  ,draw opacity=1 ]   (381,140) -- (409,140) ;
\draw [shift={(411,140)}, rotate = 180] [color={rgb, 255:red, 155; green, 155; blue, 155 }  ,draw opacity=1 ][line width=0.75]    (6.56,-1.97) .. controls (4.17,-0.84) and (1.99,-0.18) .. (0,0) .. controls (1.99,0.18) and (4.17,0.84) .. (6.56,1.97)   ;
%Straight Lines [id:da5885887633516773] 
\draw [color={rgb, 255:red, 155; green, 155; blue, 155 }  ,draw opacity=1 ]   (381,136) -- (381,144) ;
%Curve Lines [id:da58316887037238] 
\draw    (469.67,100) .. controls (471.47,108.2) and (475.87,115.8) .. (488.67,121) ;
%Curve Lines [id:da5892164363370944] 
\draw    (509.67,100) .. controls (509.47,126.2) and (469.87,121.8) .. (469.67,180) ;
%Curve Lines [id:da37395946502444233] 
\draw    (496.67,127) .. controls (508.27,145.4) and (511.07,156) .. (509.67,180) ;
%Curve Lines [id:da857579069977218] 
\draw    (419.67,100) .. controls (421.47,108.2) and (421.2,139.8) .. (434,145) ;
%Curve Lines [id:da8117026709461808] 
\draw    (459.67,100) .. controls (459.8,150.6) and (420.53,153.2) .. (420,180) ;
%Curve Lines [id:da2598178790696365] 
\draw    (446,150) .. controls (457.4,155.8) and (459.73,172) .. (460,180) ;




\end{tikzpicture}
    \end{center}
\end{frame}
\begin{frame}
    \begin{center}
        %!TEX root = ./main.tex



\tikzset{every picture/.style={line width=0.75pt}} %set default line width to 0.75pt        

\begin{tikzpicture}[x=0.75pt,y=0.75pt,yscale=-1.3,xscale=1.3]
%uncomment if require: \path (0,300); %set diagram left start at 0, and has height of 300

%Straight Lines [id:da021068853778946184] 
\draw [color={rgb, 255:red, 155; green, 155; blue, 155 }  ,draw opacity=1 ]   (170,146) -- (198,146) ;
\draw [shift={(200,146)}, rotate = 180] [color={rgb, 255:red, 155; green, 155; blue, 155 }  ,draw opacity=1 ][line width=0.75]    (6.56,-1.97) .. controls (4.17,-0.84) and (1.99,-0.18) .. (0,0) .. controls (1.99,0.18) and (4.17,0.84) .. (6.56,1.97)   ;
%Straight Lines [id:da95151298393968] 
\draw [color={rgb, 255:red, 155; green, 155; blue, 155 }  ,draw opacity=1 ]   (170,142) -- (170,150) ;
%Curve Lines [id:da267445420900821] 
\draw    (70,70) .. controls (70.2,109) and (110.2,110.2) .. (110,140) ;
%Curve Lines [id:da9560126881859321] 
\draw    (110,70) .. controls (109.4,92.6) and (104.8,103.4) .. (96,109) ;
%Curve Lines [id:da69932409686339] 
\draw    (70,140) .. controls (74.6,123.4) and (84.4,116.2) .. (88,112) ;
%Straight Lines [id:da3984579751574062] 
\draw    (130,70) -- (130,140) ;
%Curve Lines [id:da9275419139907757] 
\draw    (90,160) .. controls (90.2,199) and (130.2,200.2) .. (130,230) ;
%Curve Lines [id:da7634162233242681] 
\draw    (130,160) .. controls (129.4,182.6) and (124.8,193.4) .. (116,199) ;
%Curve Lines [id:da3594253594620098] 
\draw    (90,230) .. controls (94.6,213.4) and (104.4,206.2) .. (108,202) ;
%Straight Lines [id:da8993733494160759] 
\draw    (70,160) -- (70,230) ;
%Curve Lines [id:da3504419970988496] 
\draw    (260,70) .. controls (260.2,92.29) and (300.2,92.97) .. (300,110) ;
%Curve Lines [id:da858985711627] 
\draw    (295,70) .. controls (294.4,82.91) and (291.6,82.6) .. (285,90.29) ;
%Curve Lines [id:da617506159593737] 
\draw    (260,110) .. controls (264.6,100.51) and (274.4,96.4) .. (278,94) ;
%Straight Lines [id:da6443868731225134] 
\draw    (320,70) -- (320,110) ;
%Curve Lines [id:da9288543035495236] 
\draw    (300,110) .. controls (300.2,126.71) and (320.2,127.23) .. (320,140) ;
%Curve Lines [id:da16812105145415968] 
\draw    (320,110) .. controls (319.4,119.69) and (318,120.8) .. (311,123.71) ;
%Curve Lines [id:da5422492381478998] 
\draw    (295,140) .. controls (298,127.8) and (302.4,127.8) .. (306,126) ;
%Straight Lines [id:da0187639766750477] 
\draw    (260,110) -- (260,140) ;
%Curve Lines [id:da7389794431084341] 
\draw    (286,160) .. controls (286.2,182.29) and (320.2,182.97) .. (320,200) ;
%Curve Lines [id:da6936020178428224] 
\draw    (320,160) .. controls (319.4,172.91) and (315.8,177.09) .. (307,180.29) ;
%Curve Lines [id:da18137766551243817] 
\draw    (287,200) .. controls (287,189.8) and (294.4,186.4) .. (298,184) ;
%Straight Lines [id:da9358369635636983] 
\draw    (260,160) -- (260,200) ;
%Curve Lines [id:da9019075794007094] 
\draw    (260,200) .. controls (260.2,216.71) and (300.2,217.23) .. (300,230) ;
%Curve Lines [id:da39127651703057165] 
\draw    (287,200) .. controls (286.6,207.4) and (286.2,211.8) .. (280,213.71) ;
%Curve Lines [id:da8806113468711441] 
\draw    (260,230) .. controls (262.2,221.4) and (267.8,217.8) .. (273,217) ;
%Straight Lines [id:da20734068021017715] 
\draw    (320,200) -- (320,230) ;

% Text Node
\draw (46,52.4) node [anchor=north west][inner sep=0.75pt]    {$D_{1}$};
% Text Node
\draw (46,142.4) node [anchor=north west][inner sep=0.75pt]    {$D_{2}$};
% Text Node
\draw (216,52.4) node [anchor=north west][inner sep=0.75pt]    {$D_{1} D_{2}$};
% Text Node
\draw (216,142.4) node [anchor=north west][inner sep=0.75pt]    {$D_{2} D_{1}$};


\end{tikzpicture}

    \end{center}
\end{frame}
\begin{frame}
    \begin{center}
        

\tikzset{every picture/.style={line width=0.75pt}} %set default line width to 0.75pt        

\begin{tikzpicture}[x=0.75pt,y=0.75pt,yscale=-1.5,xscale=1.5]
%uncomment if require: \path (0,300); %set diagram left start at 0, and has height of 300

%Straight Lines [id:da021068853778946184] 
\draw [color={rgb, 255:red, 155; green, 155; blue, 155 }  ,draw opacity=1 ]   (80,120) -- (108,120) ;
\draw [shift={(110,120)}, rotate = 180] [color={rgb, 255:red, 155; green, 155; blue, 155 }  ,draw opacity=1 ][line width=0.75]    (6.56,-1.97) .. controls (4.17,-0.84) and (1.99,-0.18) .. (0,0) .. controls (1.99,0.18) and (4.17,0.84) .. (6.56,1.97)   ;
%Straight Lines [id:da95151298393968] 
\draw [color={rgb, 255:red, 155; green, 155; blue, 155 }  ,draw opacity=1 ]   (80,116) -- (80,124) ;
%Straight Lines [id:da3105397663180821] 
\draw    (30,80) -- (30,160) ;
%Straight Lines [id:da41257341713042317] 
\draw    (60,80) -- (60,160) ;
%Curve Lines [id:da11900760040973013] 
\draw    (150,120) .. controls (149.8,140.6) and (120.2,139.8) .. (120,160) ;
%Curve Lines [id:da21344602026956538] 
\draw    (120,80) .. controls (121,102.2) and (149.4,98.6) .. (150,120) ;
%Curve Lines [id:da9689941604419631] 
\draw    (150,80) .. controls (149.8,93.8) and (148.2,95.8) .. (140,100) ;
%Curve Lines [id:da6960165665741206] 
\draw    (133,103) .. controls (122.6,107.8) and (109.8,125.8) .. (132,138) ;
%Curve Lines [id:da16812588273365814] 
\draw    (139,142) .. controls (147.4,143.8) and (150.2,151) .. (150,160) ;
%Straight Lines [id:da8993877656078348] 
\draw [color={rgb, 255:red, 155; green, 155; blue, 155 }  ,draw opacity=1 ]   (259.99,120) -- (287.99,120) ;
\draw [shift={(289.99,120)}, rotate = 180] [color={rgb, 255:red, 155; green, 155; blue, 155 }  ,draw opacity=1 ][line width=0.75]    (6.56,-1.97) .. controls (4.17,-0.84) and (1.99,-0.18) .. (0,0) .. controls (1.99,0.18) and (4.17,0.84) .. (6.56,1.97)   ;
%Straight Lines [id:da3147593233270113] 
\draw [color={rgb, 255:red, 155; green, 155; blue, 155 }  ,draw opacity=1 ]   (259.99,116) -- (259.99,124) ;
%Straight Lines [id:da36120199173040735] 
\draw    (209.99,80) -- (209.99,160) ;
%Straight Lines [id:da767115805534779] 
\draw    (239.99,80) -- (239.99,160) ;
%Curve Lines [id:da44641148036727185] 
\draw    (300,120) .. controls (299.8,140.6) and (330.2,139.8) .. (330,160) ;
%Curve Lines [id:da6320743672082285] 
\draw    (330,80) .. controls (331,102.2) and (299.4,98.6) .. (300,120) ;
%Curve Lines [id:da6313849027023718] 
\draw    (311,142) .. controls (307.2,146.2) and (300.2,146.6) .. (300,160) ;
%Curve Lines [id:da0589500737742894] 
\draw    (319,102) .. controls (339.4,117.4) and (325.8,130.2) .. (318,137) ;
%Curve Lines [id:da7255177605237971] 
\draw    (310,99) .. controls (305.4,96.6) and (302.2,94.6) .. (300,80) ;
%Curve Lines [id:da23638349304952366] 
\draw    (149.67,180) .. controls (149.47,211) and (110.27,227.4) .. (109.67,250) ;
%Curve Lines [id:da8983550529947678] 
\draw    (129.67,180) .. controls (129.47,197) and (110.27,196.6) .. (109.67,210) ;
%Curve Lines [id:da5344052460604444] 
\draw    (109.67,210) .. controls (109.07,221.4) and (110.27,221.4) .. (116.67,228) ;
%Curve Lines [id:da9058141972471662] 
\draw    (129.67,250) .. controls (129.47,241) and (131.47,239.4) .. (123.67,232) ;
%Curve Lines [id:da1999229819183379] 
\draw    (109.67,180) .. controls (109.87,190.6) and (113.47,189.8) .. (119.67,194) ;
%Curve Lines [id:da5382611206991912] 
\draw    (123.67,197) .. controls (130.27,200.2) and (131.07,201.4) .. (135.67,207) ;
%Curve Lines [id:da9383909926867282] 
\draw    (139.67,212) .. controls (144.27,219.4) and (151.87,225) .. (149.67,250) ;
%Straight Lines [id:da9655390534837395] 
\draw [color={rgb, 255:red, 155; green, 155; blue, 155 }  ,draw opacity=1 ]   (169.67,216) -- (197.67,216) ;
\draw [shift={(199.67,216)}, rotate = 180] [color={rgb, 255:red, 155; green, 155; blue, 155 }  ,draw opacity=1 ][line width=0.75]    (6.56,-1.97) .. controls (4.17,-0.84) and (1.99,-0.18) .. (0,0) .. controls (1.99,0.18) and (4.17,0.84) .. (6.56,1.97)   ;
%Straight Lines [id:da030621442854791958] 
\draw [color={rgb, 255:red, 155; green, 155; blue, 155 }  ,draw opacity=1 ]   (169.67,212) -- (169.67,220) ;
%Curve Lines [id:da5794729440142999] 
\draw    (249.67,180) .. controls (249.87,211.8) and (209.47,203) .. (209.67,250) ;
%Curve Lines [id:da10826159713872396] 
\draw    (229.67,180) .. controls (230.27,190.6) and (228.27,192.2) .. (237.67,200) ;
%Curve Lines [id:da991680044584273] 
\draw    (229.67,250) .. controls (229.87,234.2) and (249.07,235) .. (249.67,220) ;
%Curve Lines [id:da7350174882368962] 
\draw    (243.67,203) .. controls (247.87,207) and (251.07,209) .. (249.67,220) ;
%Curve Lines [id:da9744338524826922] 
\draw    (209.67,180) .. controls (211.87,196.2) and (209.87,201.6) .. (221.67,211) ;
%Curve Lines [id:da1740724973590062] 
\draw    (243.67,235) .. controls (249.67,238.4) and (248.27,243.4) .. (249.67,250) ;
%Curve Lines [id:da3320368868474375] 
\draw    (225.67,217) .. controls (232.47,226.8) and (235.47,226.6) .. (238.67,231) ;

% Text Node
\draw (85,98.4) node [anchor=north west][inner sep=0.75pt]    {$\Omega _{2}$};
% Text Node
\draw (264.99,98.4) node [anchor=north west][inner sep=0.75pt]    {$\Omega _{2}$};
% Text Node
\draw (174.67,194.4) node [anchor=north west][inner sep=0.75pt]    {$\Omega _{3}$};


\end{tikzpicture}
    \end{center}
\end{frame}
\begin{frame}{Diagramas y Trenzas Geometricas}
    \begin{block}{Definicion}
        Dados dos diagramas $D$ y $D^\prime$ decimos que son $R$-equivalentes si $D$ se puede transformas por medio de una secuencia finita de isotopias y movimientos de Reidemeister a $D^\prime$
    \end{block}
    \begin{block}{Teorema}
      Dos diagramas de trenzas representan trenzas geométricas isotopicas si y solo si estos diagramas son $R$-equivalentes.  
    \end{block}
\end{frame}
\begin{frame}{}
    \begin{block}{Teorema}
        $\mathcal{B}_n$ es un grupo.                                                            
    \end{block}
    
\end{frame}
\begin{frame}
    \begin{center}
        

\tikzset{every picture/.style={line width=0.75pt}} %set default line width to 0.75pt        

\begin{tikzpicture}[x=0.75pt,y=0.75pt,yscale=-0.7,xscale=0.7]
%uncomment if require: \path (0,877); %set diagram left start at 0, and has height of 877

%Curve Lines [id:da427260159128515] 
\draw    (62.05,127) .. controls (62.25,145.66) and (32.25,140.65) .. (32.05,159.5) ;
%Straight Lines [id:da669144458368129] 
\draw    (92.05,127) -- (92.05,159.5) ;
%Curve Lines [id:da5875226797100279] 
\draw    (32.05,127) .. controls (31.45,139.72) and (36.65,138.61) .. (44.05,142.32) ;
%Curve Lines [id:da8023235603797557] 
\draw    (50.05,144.64) .. controls (59.45,148.08) and (61.85,148.64) .. (62.05,159.5) ;
%Curve Lines [id:da9691807928909935] 
\draw    (92.09,159.5) .. controls (92.29,178.16) and (62.29,173.15) .. (62.09,192) ;
%Straight Lines [id:da8189246846133199] 
\draw    (32.05,159.5) -- (32.05,192) ;
%Curve Lines [id:da03334011193119202] 
\draw    (62.09,159.5) .. controls (61.49,172.22) and (66.69,171.11) .. (74.09,174.82) ;
%Curve Lines [id:da9703929579231297] 
\draw    (80.09,177.14) .. controls (89.49,180.58) and (91.89,181.14) .. (92.09,192) ;
%Curve Lines [id:da3476578065648639] 
\draw    (62.09,192) .. controls (62.29,210.66) and (32.29,205.65) .. (32.09,224.5) ;
%Straight Lines [id:da6047039693312939] 
\draw    (92.09,192) -- (92.09,224.5) ;
%Curve Lines [id:da9132235245863184] 
\draw    (32.09,192) .. controls (31.49,204.72) and (36.69,203.61) .. (44.09,207.32) ;
%Curve Lines [id:da7963348295748283] 
\draw    (50.09,209.64) .. controls (59.49,213.08) and (61.89,213.64) .. (62.09,224.5) ;
%Curve Lines [id:da11214896674466468] 
\draw    (92,224.5) .. controls (92.2,254.65) and (62.2,246.55) .. (62,277) ;
%Straight Lines [id:da5785722857796761] 
\draw    (32,224.5) -- (32,277) ;
%Curve Lines [id:da4563642723889444] 
\draw    (62,224.5) .. controls (61.4,245.05) and (66.6,243.25) .. (74,249.25) ;
%Curve Lines [id:da011566386673468099] 
\draw    (80,253) .. controls (89.4,258.55) and (91.8,259.45) .. (92,277) ;
%Curve Lines [id:da9127778046127226] 
\draw    (182,127) .. controls (182.2,148.54) and (212.2,142.75) .. (212,164.5) ;
%Straight Lines [id:da008174707632187195] 
\draw    (152,127) -- (152,164.5) ;
%Curve Lines [id:da45462127023717136] 
\draw    (212,127) .. controls (211.4,141.68) and (207.8,140.82) .. (201,144.68) ;
%Curve Lines [id:da9247943883726462] 
\draw    (194,147.36) .. controls (185,149.61) and (181.8,151.96) .. (182,164.5) ;
%Curve Lines [id:da304291960900533] 
\draw    (152.01,164.5) .. controls (152.21,186.04) and (182.21,180.25) .. (182.01,202) ;
%Straight Lines [id:da9527368578332924] 
\draw    (212.01,164.5) -- (212.01,202) ;
%Curve Lines [id:da12684543359533484] 
\draw    (182.01,164.5) .. controls (181.41,179.18) and (177.81,178.32) .. (171.01,182.18) ;
%Curve Lines [id:da6114193569445832] 
\draw    (164.01,184.86) .. controls (155.01,187.11) and (151.81,189.46) .. (152.01,202) ;
%Curve Lines [id:da010831733150057254] 
\draw    (182.01,202) .. controls (182.21,223.54) and (212.21,217.75) .. (212.01,239.5) ;
%Straight Lines [id:da03536115402034823] 
\draw    (152,202) -- (152,239.5) ;
%Curve Lines [id:da6439072709175471] 
\draw    (212.01,202) .. controls (211.41,216.68) and (207.81,215.82) .. (201.01,219.68) ;
%Curve Lines [id:da23875994271201506] 
\draw    (194.01,222.36) .. controls (185.01,224.61) and (181.81,226.96) .. (182.01,239.5) ;
%Curve Lines [id:da8697709322375486] 
\draw    (152.01,239.5) .. controls (152.21,261.04) and (182.21,255.25) .. (182.01,277) ;
%Straight Lines [id:da6076361908657307] 
\draw    (212.01,239.5) -- (212.01,277) ;
%Curve Lines [id:da47015848638541236] 
\draw    (182.01,239.5) .. controls (181.41,254.18) and (177.81,253.32) .. (171.01,257.18) ;
%Curve Lines [id:da3578205966770538] 
\draw    (164.01,259.86) .. controls (155.01,262.11) and (151.81,264.46) .. (152.01,277) ;
%Curve Lines [id:da531599854802458] 
\draw    (299.81,50.39) .. controls (311.63,50.12) and (308.61,79.35) .. (320.56,79.46) ;
%Straight Lines [id:da7115570026778073] 
\draw    (299.65,21.18) -- (320.23,21.04) ;
%Curve Lines [id:da20874798824463314] 
\draw    (299.97,79.6) .. controls (308.03,80.13) and (307.3,75.07) .. (309.61,67.85) ;
%Curve Lines [id:da16452737584436772] 
\draw    (311.05,62) .. controls (313.17,52.83) and (313.51,50.49) .. (320.4,50.25) ;
%Curve Lines [id:da0001267340971137232] 
\draw    (320.23,21) .. controls (332.06,20.73) and (329.04,49.96) .. (340.98,50.08) ;
%Straight Lines [id:da7580512650775525] 
\draw    (320.56,79.46) -- (341.14,79.33) ;
%Curve Lines [id:da9957192227611716] 
\draw    (320.39,50.21) .. controls (328.46,50.74) and (327.72,45.68) .. (330.04,38.46) ;
%Curve Lines [id:da2752293919286407] 
\draw    (331.47,32.61) .. controls (333.6,23.45) and (333.94,21.11) .. (340.82,20.87) ;
%Curve Lines [id:da4248810803242654] 
\draw    (340.98,50.08) .. controls (352.8,49.81) and (349.79,79.04) .. (361.73,79.16) ;
%Straight Lines [id:da8504788815198621] 
\draw    (340.82,20.87) -- (361.41,20.74) ;
%Curve Lines [id:da9144139734174279] 
\draw    (341.14,79.29) .. controls (349.2,79.82) and (348.47,74.76) .. (350.78,67.54) ;
%Curve Lines [id:da5788507049444858] 
\draw    (352.22,61.69) .. controls (354.35,52.52) and (354.69,50.18) .. (361.57,49.95) ;
%Curve Lines [id:da24991551123604183] 
\draw    (361.41,20.83) .. controls (380.5,20.51) and (375.53,49.75) .. (394.82,49.83) ;
%Straight Lines [id:da2279207192126409] 
\draw    (361.73,79.25) -- (394.98,79.04) ;
%Curve Lines [id:da869940382936184] 
\draw    (361.57,50.04) .. controls (374.59,50.54) and (373.42,45.48) .. (377.18,38.25) ;
%Curve Lines [id:da4641619409636202] 
\draw    (379.52,32.4) .. controls (382.99,23.22) and (383.55,20.88) .. (394.66,20.62) ;
%Curve Lines [id:da9490813438634549] 
\draw    (394.82,49.83) .. controls (408.46,49.54) and (404.64,20.36) .. (418.42,20.46) ;
%Straight Lines [id:da68328132238534] 
\draw    (394.98,79.04) -- (418.74,78.88) ;
%Curve Lines [id:da12353792960760934] 
\draw    (394.66,20.62) .. controls (403.96,21.14) and (403.44,24.65) .. (405.92,31.26) ;
%Curve Lines [id:da6705891019087905] 
\draw    (407.65,38.06) .. controls (409.13,46.81) and (410.64,49.92) .. (418.58,49.67) ;
%Curve Lines [id:da9239147121814012] 
\draw    (418.74,78.88) .. controls (432.38,78.59) and (428.55,49.41) .. (442.33,49.51) ;
%Straight Lines [id:da18931787803211197] 
\draw    (418.42,20.46) -- (442.17,20.3) ;
%Curve Lines [id:da29055757381703284] 
\draw    (418.58,49.67) .. controls (427.88,50.19) and (427.36,53.7) .. (429.83,60.3) ;
%Curve Lines [id:da5425206352743233] 
\draw    (431.57,67.11) .. controls (433.04,75.86) and (434.55,78.97) .. (442.49,78.72) ;
%Curve Lines [id:da41494150959375964] 
\draw    (442.33,49.51) .. controls (455.97,49.23) and (452.15,20.04) .. (465.92,20.15) ;
%Straight Lines [id:da132979245306439] 
\draw    (442.49,78.73) -- (466.25,78.58) ;
%Curve Lines [id:da5680042325328389] 
\draw    (442.17,20.3) .. controls (451.47,20.83) and (450.95,24.34) .. (453.43,30.94) ;
%Curve Lines [id:da6672133326873362] 
\draw    (455.16,37.75) .. controls (456.64,46.5) and (458.15,49.61) .. (466.09,49.36) ;
%Curve Lines [id:da5764858313819528] 
\draw    (466.25,78.57) .. controls (479.89,78.29) and (476.06,49.1) .. (489.84,49.21) ;
%Straight Lines [id:da9437582235565463] 
\draw    (465.92,20.15) -- (489.68,20) ;
%Curve Lines [id:da3423748266275308] 
\draw    (466.09,49.36) .. controls (475.39,49.89) and (474.86,53.4) .. (477.34,60) ;
%Curve Lines [id:da20080394546758273] 
\draw    (479.08,66.81) .. controls (480.55,75.56) and (482.06,78.67) .. (490,78.42) ;
%Straight Lines [id:da9678508680287174] 
\draw    (230,200) -- (278,200) ;
\draw [shift={(280,200)}, rotate = 180] [color={rgb, 255:red, 0; green, 0; blue, 0 }  ][line width=0.75]    (10.93,-3.29) .. controls (6.95,-1.4) and (3.31,-0.3) .. (0,0) .. controls (3.31,0.3) and (6.95,1.4) .. (10.93,3.29)   ;
%Curve Lines [id:da7308439505195594] 
\draw    (299.81,128.28) .. controls (311.63,128.01) and (308.61,157.24) .. (320.56,157.36) ;
%Straight Lines [id:da846513170034754] 
\draw    (299.65,99.07) -- (320.23,98.94) ;
%Curve Lines [id:da8111277427735796] 
\draw    (299.97,157.49) .. controls (308.03,158.02) and (307.3,152.97) .. (309.61,145.74) ;
%Curve Lines [id:da42089992613742544] 
\draw    (311.05,139.89) .. controls (313.17,130.73) and (313.51,128.39) .. (320.4,128.15) ;
%Curve Lines [id:da8563546535375895] 
\draw    (320.23,98.89) .. controls (332.06,98.62) and (329.04,127.85) .. (340.98,127.97) ;
%Straight Lines [id:da4667452310801803] 
\draw    (320.56,157.36) -- (341.14,157.23) ;
%Curve Lines [id:da40900442234538137] 
\draw    (320.39,128.1) .. controls (328.46,128.64) and (327.72,123.58) .. (330.04,116.36) ;
%Curve Lines [id:da2505747526551777] 
\draw    (331.47,110.51) .. controls (333.6,101.34) and (333.94,99) .. (340.82,98.76) ;
%Curve Lines [id:da1318615718347721] 
\draw    (340.98,127.97) .. controls (352.8,127.7) and (349.79,156.93) .. (361.73,157.05) ;
%Straight Lines [id:da33122994992169363] 
\draw    (340.82,98.76) -- (361.41,98.63) ;
%Curve Lines [id:da4392684723589957] 
\draw    (341.14,157.18) .. controls (349.2,157.71) and (348.47,152.66) .. (350.78,145.44) ;
%Curve Lines [id:da14605677214321222] 
\draw    (352.22,139.58) .. controls (354.35,130.42) and (354.69,128.08) .. (361.57,127.84) ;
%Straight Lines [id:da3422396945440085] 
\draw    (361.73,157.14) -- (394.98,156.93) ;
%Straight Lines [id:da5773890702994031] 
\draw    (394.98,156.93) -- (418.74,156.78) ;
%Curve Lines [id:da9252240206443388] 
\draw    (418.74,156.77) .. controls (432.38,156.49) and (428.55,127.3) .. (442.33,127.41) ;
%Straight Lines [id:da08142014538727405] 
\draw    (418.42,98.35) -- (442.17,98.2) ;
%Curve Lines [id:da6043299896294531] 
\draw    (418.58,127.56) .. controls (427.88,128.08) and (427.36,131.59) .. (429.83,138.2) ;
%Curve Lines [id:da857933680416637] 
\draw    (431.57,145) .. controls (433.04,153.76) and (434.55,156.86) .. (442.49,156.62) ;
%Curve Lines [id:da558444715983223] 
\draw    (442.33,127.41) .. controls (455.97,127.13) and (452.15,97.94) .. (465.92,98.05) ;
%Straight Lines [id:da0010087886285466974] 
\draw    (442.49,156.63) -- (466.25,156.47) ;
%Curve Lines [id:da6558981134377307] 
\draw    (442.17,98.2) .. controls (451.47,98.72) and (450.95,102.23) .. (453.43,108.84) ;
%Curve Lines [id:da20262856244113558] 
\draw    (455.16,115.64) .. controls (456.64,124.4) and (458.15,127.5) .. (466.09,127.26) ;
%Curve Lines [id:da9402828684572856] 
\draw    (466.25,156.47) .. controls (479.89,156.18) and (476.06,127) .. (489.84,127.1) ;
%Straight Lines [id:da0655025037371203] 
\draw    (465.92,98.05) -- (489.68,97.89) ;
%Curve Lines [id:da276379066843348] 
\draw    (466.09,127.26) .. controls (475.39,127.78) and (474.86,131.29) .. (477.34,137.9) ;
%Curve Lines [id:da02073947493596051] 
\draw    (479.08,144.7) .. controls (480.55,153.45) and (482.06,156.56) .. (490,156.31) ;
%Straight Lines [id:da7950315894943095] 
\draw    (361.41,98.63) -- (418.42,98.35) ;
%Straight Lines [id:da45172338628325837] 
\draw    (361.57,127.84) -- (418.58,127.56) ;
%Curve Lines [id:da8320898145842732] 
\draw    (300.16,206.18) .. controls (311.98,205.91) and (308.97,235.14) .. (320.91,235.25) ;
%Straight Lines [id:da2548387102144052] 
\draw    (300,176.97) -- (320.59,176.83) ;
%Curve Lines [id:da409883616871649] 
\draw    (300.32,235.39) .. controls (308.38,235.92) and (307.65,230.86) .. (309.96,223.64) ;
%Curve Lines [id:da8710039812889714] 
\draw    (311.4,217.79) .. controls (313.53,208.62) and (313.87,206.28) .. (320.75,206.04) ;
%Curve Lines [id:da4884793512039546] 
\draw    (320.59,176.79) .. controls (332.41,176.52) and (329.39,205.75) .. (341.33,205.87) ;
%Straight Lines [id:da4873947944494157] 
\draw    (320.91,235.25) -- (341.5,235.12) ;
%Curve Lines [id:da13237144060625328] 
\draw    (320.75,206) .. controls (328.81,206.53) and (328.08,201.47) .. (330.39,194.25) ;
%Curve Lines [id:da5907249199798268] 
\draw    (331.83,188.4) .. controls (333.95,179.23) and (334.29,176.9) .. (341.17,176.66) ;
%Straight Lines [id:da14164508530301012] 
\draw    (341.17,176.66) -- (361.76,176.53) ;
%Straight Lines [id:da8889099653035093] 
\draw    (341.5,235.12) -- (395.34,234.82) ;
%Straight Lines [id:da8237680629290057] 
\draw    (395.34,234.82) -- (442.85,234.52) ;
%Straight Lines [id:da5430475777888927] 
\draw    (418.77,176.24) -- (442.52,176.09) ;
%Curve Lines [id:da48506072461021554] 
\draw    (442.68,205.3) .. controls (456.32,205.02) and (452.5,175.83) .. (466.28,175.94) ;
%Straight Lines [id:da5608790679234237] 
\draw    (442.85,234.52) -- (466.6,234.37) ;
%Curve Lines [id:da894534360023443] 
\draw    (442.52,176.09) .. controls (451.82,176.62) and (451.3,180.13) .. (453.78,186.73) ;
%Curve Lines [id:da30195238061608853] 
\draw    (455.51,193.54) .. controls (456.99,202.29) and (458.5,205.4) .. (466.44,205.15) ;
%Curve Lines [id:da34717666389077595] 
\draw    (466.6,234.36) .. controls (480.24,234.08) and (476.41,204.89) .. (490.19,205) ;
%Straight Lines [id:da5707306301760324] 
\draw    (466.28,175.94) -- (490.03,175.79) ;
%Curve Lines [id:da39489551630013087] 
\draw    (466.44,205.15) .. controls (475.74,205.68) and (475.22,209.18) .. (477.69,215.79) ;
%Curve Lines [id:da32809801630168434] 
\draw    (479.43,222.59) .. controls (480.9,231.35) and (482.41,234.45) .. (490.35,234.21) ;
%Straight Lines [id:da78774750197805] 
\draw    (361.76,176.53) -- (418.77,176.24) ;
%Straight Lines [id:da5555019473591153] 
\draw    (341.33,205.87) -- (442.68,205.3) ;
%Curve Lines [id:da10634566902173404] 
\draw    (300.16,282.86) .. controls (311.98,282.59) and (308.97,311.82) .. (320.91,311.94) ;
%Straight Lines [id:da8074961733025664] 
\draw    (300,253.65) -- (320.59,253.52) ;
%Curve Lines [id:da5307906057585656] 
\draw    (300.32,312.07) .. controls (308.38,312.6) and (307.65,307.54) .. (309.96,300.32) ;
%Curve Lines [id:da21717506589102253] 
\draw    (311.4,294.47) .. controls (313.53,285.3) and (313.87,282.96) .. (320.75,282.73) ;
%Straight Lines [id:da43315198352162987] 
\draw    (320.91,311.94) -- (341.5,311.8) ;
%Straight Lines [id:da8450562127092562] 
\draw    (341.5,311.8) -- (395.34,311.51) ;
%Straight Lines [id:da2743124271814852] 
\draw    (395.34,311.51) -- (442.85,311.2) ;
%Straight Lines [id:da32793093566058207] 
\draw    (442.85,311.2) -- (466.6,311.05) ;
%Curve Lines [id:da084899653525769] 
\draw    (466.6,311.04) .. controls (480.24,310.76) and (476.41,281.57) .. (490.19,281.68) ;
%Straight Lines [id:da42665139226076687] 
\draw    (466.28,252.62) -- (490.03,252.47) ;
%Curve Lines [id:da5635641705148144] 
\draw    (466.44,281.83) .. controls (475.74,282.36) and (475.22,285.87) .. (477.69,292.47) ;
%Curve Lines [id:da8266294220833964] 
\draw    (479.43,299.28) .. controls (480.9,308.03) and (482.41,311.14) .. (490.35,310.89) ;
%Straight Lines [id:da3899148707146438] 
\draw    (320.59,253.52) -- (466.28,252.62) ;
%Straight Lines [id:da2097973018749848] 
\draw    (320.75,282.73) -- (466.44,281.83) ;
%Straight Lines [id:da7345031013299305] 
\draw    (300,331.58) -- (490,331.58) ;
%Straight Lines [id:da5997175761510622] 
\draw    (300,360.79) -- (490,360.79) ;
%Straight Lines [id:da924695380698776] 
\draw    (300,390) -- (490,390) ;

% Text Node
\draw (23,102.4) node [anchor=north west][inner sep=0.75pt]    {$\sigma _{1}^{+} \sigma _{2}^{+} \sigma _{1}^{+} \sigma _{2}^{+}$};
% Text Node
\draw (143,102.4) node [anchor=north west][inner sep=0.75pt]    {$\sigma _{2}^{-} \sigma _{1}^{-} \sigma _{2}^{-} \sigma _{1}^{-}$};
% Text Node
\draw (380,79.58) node [anchor=north west][inner sep=0.75pt]    {$\downarrow $};
% Text Node
\draw (380,157.48) node [anchor=north west][inner sep=0.75pt]    {$\downarrow $};
% Text Node
\draw (380,234.4) node [anchor=north west][inner sep=0.75pt]    {$\downarrow $};
% Text Node
\draw (380,312.29) node [anchor=north west][inner sep=0.75pt]    {$\downarrow $};


\end{tikzpicture}

    \end{center}
\end{frame}
\begin{frame}{Grupo de Trenzas de Artin $B_n$}

\begin{block}{Definición}
    El grupo de trenzas de Artin $B_n$ es el grupo generado por $n-1$ generadores $\sigma_i$ con $i=1,2,\ldots n-1$ y las relaciones
    $$\sigma_i\sigma_j=\sigma_j\sigma_i,$$
    para todo $i,j=1,2,\ldots n-1$ con $|i-j|\geq 2,$ y
    $$\sigma_i\sigma_{i+1}\sigma_i=\sigma_{i+1}\sigma_i\sigma_{i+1},$$
    para $i=1,2,\ldots n-2.$
\end{block}
\end{frame}
\begin{frame}{}
    \begin{block}{Lema}
        Si existen elementos $\{s_i|i=1,\ldots,n-1\}$ en un grupo $G$ tales que se satisfacen las relaciones de trenzas, entonces existe un unico homomorfismo de grupos $f:B_n\to G$, tal que $s_i=f(\sigma_i).$
    \end{block}
    \begin{block}{Lema}
         Las trasposiciones simples $s_i=(i\,i+1)\in S_n$, generan $S_n$ y satisfacen las relaciones de trenzas.
    \end{block}
    \begin{block}{Teorema}
        El grupo $B_n$ para $n\geq 3$ no es abeliano
    \end{block}
\end{frame}
\begin{frame}{Equivalencia entre $\mathcal{B}_n$ y $B_n$}
    \begin{block}{Proposicion}
        Los elementos $\sigma_1^+,\ldots,\sigma_{n-1}^+\in \mathcal{B}_n$ satisfacen las relaciones de trenzas. 
    \end{block}
    \begin{block}{Teorema}
        Para $\varepsilon=\pm,$ existe un unico homomorfismo $\varphi_{\varepsilon}:B_n\to\mathcal{B}_n$ tal que \\$\varphi_{\varepsilon}(\sigma_i)=\sigma_i^{\varepsilon}$ para todo $i=1,\ldots n-1.$ Ademas el homorfismo $\varphi_\varepsilon$ resulta ser un isormorfismo. 
    \end{block}
\end{frame}
\begin{frame}
    \begin{center}
        
    

\tikzset{every picture/.style={line width=0.75pt}} %set default line width to 0.75pt        

\begin{tikzpicture}[x=0.75pt,y=0.75pt,yscale=-1.5,xscale=1.5]
%uncomment if require: \path (0,300); %set diagram left start at 0, and has height of 300

%Straight Lines [id:da43485227266233983] 
\draw    (170,20) -- (170,120) ;
%Straight Lines [id:da9331696760365621] 
\draw    (210,20) -- (210,120) ;
%Straight Lines [id:da9617020865382828] 
\draw    (300,20) -- (300,120) ;
%Straight Lines [id:da23545398468453227] 
\draw    (340,20) -- (340,120) ;
%Curve Lines [id:da7964129888127446] 
\draw    (240,120) .. controls (240.6,69.8) and (269.8,69.8) .. (270,20) ;
%Curve Lines [id:da9092522991774518] 
\draw    (240,20) .. controls (242.2,52.6) and (244.6,60.6) .. (251,67) ;
%Curve Lines [id:da14137481359772275] 
\draw    (259,73) .. controls (263.6,82.6) and (270.2,92.6) .. (270,120) ;
%Straight Lines [id:da7700112899007484] 
\draw    (170,149) -- (170,249) ;
%Straight Lines [id:da5513354586153557] 
\draw    (210,149) -- (210,249) ;
%Straight Lines [id:da6374072931177768] 
\draw    (300,149) -- (300,249) ;
%Straight Lines [id:da704810691253697] 
\draw    (340,149) -- (340,249) ;
%Curve Lines [id:da20368613461263263] 
\draw    (270,249) .. controls (270.6,198.8) and (239.8,199.8) .. (240,150) ;
%Curve Lines [id:da48874246949849454] 
\draw    (270,150) .. controls (270.2,182.6) and (262.4,187.2) .. (257,195) ;
%Curve Lines [id:da2656325163455132] 
\draw    (251,200) .. controls (243.2,207) and (240.2,222.6) .. (240,250) ;

% Text Node
\draw (176,59.4) node [anchor=north west][inner sep=0.75pt]    {$\cdots $};
% Text Node
\draw (306,60.4) node [anchor=north west][inner sep=0.75pt]    {$\cdots $};
% Text Node
\draw (166,5.4) node [anchor=north west][inner sep=0.75pt]  [font=\scriptsize]  {$1$};
% Text Node
\draw (199,6.4) node [anchor=north west][inner sep=0.75pt]  [font=\scriptsize]  {$i-1$};
% Text Node
\draw (237,6.4) node [anchor=north west][inner sep=0.75pt]  [font=\scriptsize]  {$i$};
% Text Node
\draw (259,6.4) node [anchor=north west][inner sep=0.75pt]  [font=\scriptsize]  {$i+1$};
% Text Node
\draw (289,6.4) node [anchor=north west][inner sep=0.75pt]  [font=\scriptsize]  {$i+2$};
% Text Node
\draw (336,5.4) node [anchor=north west][inner sep=0.75pt]  [font=\scriptsize]  {$n$};
% Text Node
\draw (138,56.4) node [anchor=north west][inner sep=0.75pt]    {$\sigma _{i}^{+}$};
% Text Node
\draw (176,188.4) node [anchor=north west][inner sep=0.75pt]    {$\cdots $};
% Text Node
\draw (306,189.4) node [anchor=north west][inner sep=0.75pt]    {$\cdots $};
% Text Node
\draw (166,134.4) node [anchor=north west][inner sep=0.75pt]  [font=\scriptsize]  {$1$};
% Text Node
\draw (199,135.4) node [anchor=north west][inner sep=0.75pt]  [font=\scriptsize]  {$i-1$};
% Text Node
\draw (237,135.4) node [anchor=north west][inner sep=0.75pt]  [font=\scriptsize]  {$i$};
% Text Node
\draw (259,135.4) node [anchor=north west][inner sep=0.75pt]  [font=\scriptsize]  {$i+1$};
% Text Node
\draw (289,135.4) node [anchor=north west][inner sep=0.75pt]  [font=\scriptsize]  {$i+2$};
% Text Node
\draw (336,134.4) node [anchor=north west][inner sep=0.75pt]  [font=\scriptsize]  {$n$};
% Text Node
\draw (138,185.4) node [anchor=north west][inner sep=0.75pt]    {$\sigma _{i}^{-}$};


\end{tikzpicture}

    \end{center}
\end{frame}
\begin{frame}
    \begin{center}
        


\tikzset{every picture/.style={line width=0.75pt}} %set default line width to 0.75pt        

\begin{tikzpicture}[x=0.75pt,y=0.75pt,yscale=-1.5,xscale=1.5]
%uncomment if require: \path (0,300); %set diagram left start at 0, and has height of 300

%Straight Lines [id:da43485227266233983] 
\draw    (191,40) -- (191,85) ;
%Straight Lines [id:da9331696760365621] 
\draw    (231,40) -- (231,85) ;
%Straight Lines [id:da9617020865382828] 
\draw    (321,40) -- (321,85) ;
%Straight Lines [id:da23545398468453227] 
\draw    (431,40) -- (431,85) ;
%Curve Lines [id:da7964129888127446] 
\draw    (261,85) .. controls (261.6,62.41) and (290.8,62.41) .. (291,40) ;
%Curve Lines [id:da9092522991774518] 
\draw    (261,40) .. controls (263.2,54.67) and (265.6,58.27) .. (272,61.15) ;
%Curve Lines [id:da14137481359772275] 
\draw    (280,63.85) .. controls (284.6,68.17) and (291.2,72.67) .. (291,85) ;
%Straight Lines [id:da2749823058854669] 
\draw    (361,40) -- (361,85) ;
%Straight Lines [id:da719071174170849] 
\draw    (391,40) -- (391,85) ;
%Straight Lines [id:da5066121581804023] 
\draw    (191,85) -- (191,130) ;
%Straight Lines [id:da07838388991515421] 
\draw    (231,85) -- (231,130) ;
%Straight Lines [id:da003494212995229007] 
\draw    (321,85) -- (321,130) ;
%Straight Lines [id:da7151909637664416] 
\draw    (431,85) -- (431,130) ;
%Curve Lines [id:da44479415108141185] 
\draw    (361,130) .. controls (361.6,107.41) and (390.8,107.41) .. (391,85) ;
%Curve Lines [id:da719812928955529] 
\draw    (361,85) .. controls (363.2,99.67) and (365.6,103.27) .. (372,106.15) ;
%Curve Lines [id:da39257749071884107] 
\draw    (380,108.85) .. controls (384.6,113.17) and (391.2,117.67) .. (391,130) ;
%Straight Lines [id:da7560399909014393] 
\draw    (261,85) -- (261,130) ;
%Straight Lines [id:da8961903157629305] 
\draw    (291,85) -- (291,130) ;
%Straight Lines [id:da7297484067582013] 
\draw    (190,225) -- (190,270) ;
%Straight Lines [id:da7214060474031848] 
\draw    (230,225) -- (230,270) ;
%Straight Lines [id:da8878059243905154] 
\draw    (320,225) -- (320,270) ;
%Curve Lines [id:da026795802691722104] 
\draw    (260,270) .. controls (260.6,247.41) and (289.8,247.41) .. (290,225) ;
%Curve Lines [id:da006615458158601206] 
\draw    (260,225) .. controls (262.2,239.67) and (264.6,243.27) .. (271,246.15) ;
%Curve Lines [id:da614603829677808] 
\draw    (279,248.85) .. controls (283.6,253.17) and (290.2,257.67) .. (290,270) ;
%Straight Lines [id:da47279845454294156] 
\draw    (360,225) -- (360,270) ;
%Straight Lines [id:da19830117682490833] 
\draw    (390,225) -- (390,270) ;
%Straight Lines [id:da18194370220855138] 
\draw    (190,180) -- (190,225) ;
%Straight Lines [id:da8930399384376863] 
\draw    (230,180) -- (230,225) ;
%Straight Lines [id:da5700918864394047] 
\draw    (320,180) -- (320,225) ;
%Curve Lines [id:da8906685106650633] 
\draw    (360,225) .. controls (360.6,202.41) and (389.8,202.41) .. (390,180) ;
%Curve Lines [id:da7652576772512178] 
\draw    (360,180) .. controls (362.2,194.67) and (364.6,198.27) .. (371,201.15) ;
%Curve Lines [id:da05916776170620042] 
\draw    (379,203.85) .. controls (383.6,208.17) and (390.2,212.67) .. (390,225) ;
%Straight Lines [id:da14909445257416631] 
\draw    (260,180) -- (260,225) ;
%Straight Lines [id:da9006770623896229] 
\draw    (290,180) -- (290,225) ;
%Straight Lines [id:da6250479219694898] 
\draw    (429,225) -- (429,270) ;
%Straight Lines [id:da20702343693218261] 
\draw    (429,180) -- (429,225) ;

% Text Node
\draw (197,53.55) node [anchor=north west][inner sep=0.75pt]    {$\cdots $};
% Text Node
\draw (327,54) node [anchor=north west][inner sep=0.75pt]    {$\cdots $};
% Text Node
\draw (187,26.4) node [anchor=north west][inner sep=0.75pt]  [font=\scriptsize]  {$1$};
% Text Node
\draw (220,26.4) node [anchor=north west][inner sep=0.75pt]  [font=\scriptsize]  {$i-1$};
% Text Node
\draw (258,26.4) node [anchor=north west][inner sep=0.75pt]  [font=\scriptsize]  {$i$};
% Text Node
\draw (280,26.4) node [anchor=north west][inner sep=0.75pt]  [font=\scriptsize]  {$i+1$};
% Text Node
\draw (310,26.4) node [anchor=north west][inner sep=0.75pt]  [font=\scriptsize]  {$i+2$};
% Text Node
\draw (427,27.4) node [anchor=north west][inner sep=0.75pt]  [font=\scriptsize]  {$n$};
% Text Node
\draw (147,67.4) node [anchor=north west][inner sep=0.75pt]    {$\sigma _{i}^{+} \sigma _{j}^{+}$};
% Text Node
\draw (396,54) node [anchor=north west][inner sep=0.75pt]    {$\cdots $};
% Text Node
\draw (358,25.4) node [anchor=north west][inner sep=0.75pt]  [font=\scriptsize]  {$j$};
% Text Node
\draw (380,26.4) node [anchor=north west][inner sep=0.75pt]  [font=\scriptsize]  {$j+1$};
% Text Node
\draw (197,98.55) node [anchor=north west][inner sep=0.75pt]    {$\cdots $};
% Text Node
\draw (327,99) node [anchor=north west][inner sep=0.75pt]    {$\cdots $};
% Text Node
\draw (396,99) node [anchor=north west][inner sep=0.75pt]    {$\cdots $};
% Text Node
\draw (196,238.55) node [anchor=north west][inner sep=0.75pt]    {$\cdots $};
% Text Node
\draw (326,239) node [anchor=north west][inner sep=0.75pt]    {$\cdots $};
% Text Node
\draw (187,166.4) node [anchor=north west][inner sep=0.75pt]  [font=\scriptsize]  {$1$};
% Text Node
\draw (220,166.4) node [anchor=north west][inner sep=0.75pt]  [font=\scriptsize]  {$i-1$};
% Text Node
\draw (258,166.4) node [anchor=north west][inner sep=0.75pt]  [font=\scriptsize]  {$i$};
% Text Node
\draw (280,166.4) node [anchor=north west][inner sep=0.75pt]  [font=\scriptsize]  {$i+1$};
% Text Node
\draw (310,166.4) node [anchor=north west][inner sep=0.75pt]  [font=\scriptsize]  {$i+2$};
% Text Node
\draw (427,167.4) node [anchor=north west][inner sep=0.75pt]  [font=\scriptsize]  {$n$};
% Text Node
\draw (395,239) node [anchor=north west][inner sep=0.75pt]    {$\cdots $};
% Text Node
\draw (358,165.4) node [anchor=north west][inner sep=0.75pt]  [font=\scriptsize]  {$j$};
% Text Node
\draw (380,166.4) node [anchor=north west][inner sep=0.75pt]  [font=\scriptsize]  {$j+1$};
% Text Node
\draw (196,193.55) node [anchor=north west][inner sep=0.75pt]    {$\cdots $};
% Text Node
\draw (326,194) node [anchor=north west][inner sep=0.75pt]    {$\cdots $};
% Text Node
\draw (395,194) node [anchor=north west][inner sep=0.75pt]    {$\cdots $};
% Text Node
\draw (146,202.4) node [anchor=north west][inner sep=0.75pt]    {$\sigma _{j}^{+} \sigma _{i}^{+}$};
% Text Node
\draw (293,137.4) node [anchor=north west][inner sep=0.75pt]    {$\updownarrow $};
% Text Node
\draw (312,140.4) node [anchor=north west][inner sep=0.75pt]  [font=\footnotesize]  {$isotopia$};


\end{tikzpicture}

    \end{center}
\end{frame}
\begin{frame}
    \begin{center}
        



\tikzset{every picture/.style={line width=0.75pt}} %set default line width to 0.75pt        

\begin{tikzpicture}[x=0.75pt,y=0.75pt,yscale=-1.5,xscale=1.5]
%uncomment if require: \path (0,335); %set diagram left start at 0, and has height of 335

%Straight Lines [id:da6051419366022395] 
\draw    (130,93.33) -- (130,126.67) ;
%Curve Lines [id:da9988810326669739] 
\draw    (230,93.33) .. controls (230,110) and (201,110) .. (200,126.67) ;
%Curve Lines [id:da22799961222723653] 
\draw    (200,93.33) .. controls (201.5,104.33) and (200.5,105.17) .. (212,109.33) ;
%Curve Lines [id:da1490781175691419] 
\draw    (219,111.67) .. controls (224.5,115.17) and (228.5,115.17) .. (230,126.67) ;
%Straight Lines [id:da8207937248086156] 
\draw    (170,93.33) -- (170,126.67) ;
%Straight Lines [id:da048365350972087495] 
\draw    (270,93.33) -- (270,126.67) ;
%Straight Lines [id:da783175719816261] 
\draw    (130,60) -- (130,93.33) ;
%Curve Lines [id:da21948128892783714] 
\draw    (200,60) .. controls (200,76.67) and (171,76.67) .. (170,93.33) ;
%Curve Lines [id:da11000036284050341] 
\draw    (170,60) .. controls (171.5,71) and (170.5,71.83) .. (182,76) ;
%Curve Lines [id:da2016637435430727] 
\draw    (189,78.33) .. controls (194.5,81.83) and (198.5,81.83) .. (200,93.33) ;
%Straight Lines [id:da03184093892635387] 
\draw    (230,60) -- (230,93.33) ;
%Straight Lines [id:da8580736965744533] 
\draw    (270,60) -- (270,93.33) ;
%Straight Lines [id:da3240339948692498] 
\draw    (130,126.67) -- (130,160) ;
%Curve Lines [id:da15803163782768093] 
\draw    (200,126.67) .. controls (200,143.33) and (171,143.33) .. (170,160) ;
%Curve Lines [id:da6791919280524208] 
\draw    (170,126.67) .. controls (171.5,137.67) and (170.5,138.5) .. (182,142.67) ;
%Curve Lines [id:da9636959373597402] 
\draw    (189,145) .. controls (194.5,148.5) and (198.5,148.5) .. (200,160) ;
%Straight Lines [id:da7737137812536472] 
\draw    (230,126.67) -- (230,160) ;
%Straight Lines [id:da7868366212512278] 
\draw    (270,126.67) -- (270,160) ;
%Straight Lines [id:da9117606281933235] 
\draw    (341,60) -- (341,93.33) ;
%Curve Lines [id:da4617900422359481] 
\draw    (441,60) .. controls (441,76.67) and (412,76.67) .. (411,93.33) ;
%Curve Lines [id:da7919077081110307] 
\draw    (411,60) .. controls (412.5,71) and (411.5,71.83) .. (423,76) ;
%Curve Lines [id:da556973599872705] 
\draw    (430,78.33) .. controls (435.5,81.83) and (439.5,81.83) .. (441,93.33) ;
%Straight Lines [id:da7614386709213076] 
\draw    (381,60) -- (381,93.33) ;
%Straight Lines [id:da03345456947894776] 
\draw    (481,60) -- (481,93.33) ;
%Straight Lines [id:da26200967481729853] 
\draw    (341,93.33) -- (341,126.67) ;
%Curve Lines [id:da09553132514781482] 
\draw    (411,93.33) .. controls (411,110) and (382,110) .. (381,126.67) ;
%Curve Lines [id:da2563544636504017] 
\draw    (381,93.33) .. controls (382.5,104.33) and (381.5,105.17) .. (393,109.33) ;
%Curve Lines [id:da13193755972733046] 
\draw    (400,111.67) .. controls (405.5,115.17) and (409.5,115.17) .. (411,126.67) ;
%Straight Lines [id:da9316135114510464] 
\draw    (441,93.33) -- (441,126.67) ;
%Straight Lines [id:da11527130856596324] 
\draw    (481,93.33) -- (481,126.67) ;
%Straight Lines [id:da3239224345124351] 
\draw    (341,126.67) -- (341,160) ;
%Curve Lines [id:da6738308771610565] 
\draw    (441,126.67) .. controls (441,143.33) and (412,143.33) .. (411,160) ;
%Curve Lines [id:da8633246586798422] 
\draw    (411,126.67) .. controls (412.5,137.67) and (411.5,138.5) .. (423,142.67) ;
%Curve Lines [id:da5645812605701747] 
\draw    (430,145) .. controls (435.5,148.5) and (439.5,148.5) .. (441,160) ;
%Straight Lines [id:da31368590787885864] 
\draw    (381,126.67) -- (381,160) ;
%Straight Lines [id:da4310704090732771] 
\draw    (481,126.67) -- (481,160) ;
%Straight Lines [id:da3326166111695229] 
\draw    (290,110) -- (328,110) ;
\draw [shift={(330,110)}, rotate = 180] [color={rgb, 255:red, 0; green, 0; blue, 0 }  ][line width=0.75]    (10.93,-3.29) .. controls (6.95,-1.4) and (3.31,-0.3) .. (0,0) .. controls (3.31,0.3) and (6.95,1.4) .. (10.93,3.29)   ;

% Text Node
\draw (134,101.4) node [anchor=north west][inner sep=0.75pt]    {$\cdots $};
% Text Node
\draw (235,101.4) node [anchor=north west][inner sep=0.75pt]    {$\cdots $};
% Text Node
\draw (134,68.07) node [anchor=north west][inner sep=0.75pt]    {$\cdots $};
% Text Node
\draw (235,68.07) node [anchor=north west][inner sep=0.75pt]    {$\cdots $};
% Text Node
\draw (134,134.73) node [anchor=north west][inner sep=0.75pt]    {$\cdots $};
% Text Node
\draw (235,134.73) node [anchor=north west][inner sep=0.75pt]    {$\cdots $};
% Text Node
\draw (345,68.07) node [anchor=north west][inner sep=0.75pt]    {$\cdots $};
% Text Node
\draw (446,68.07) node [anchor=north west][inner sep=0.75pt]    {$\cdots $};
% Text Node
\draw (345,101.4) node [anchor=north west][inner sep=0.75pt]    {$\cdots $};
% Text Node
\draw (446,101.4) node [anchor=north west][inner sep=0.75pt]    {$\cdots $};
% Text Node
\draw (345,134.73) node [anchor=north west][inner sep=0.75pt]    {$\cdots $};
% Text Node
\draw (446,134.73) node [anchor=north west][inner sep=0.75pt]    {$\cdots $};
% Text Node
\draw (126,46.4) node [anchor=north west][inner sep=0.75pt]  [font=\scriptsize]  {$1$};
% Text Node
\draw (167,46.4) node [anchor=north west][inner sep=0.75pt]  [font=\scriptsize]  {$i$};
% Text Node
\draw (189,46.4) node [anchor=north west][inner sep=0.75pt]  [font=\scriptsize]  {$i+1$};
% Text Node
\draw (220,46.4) node [anchor=north west][inner sep=0.75pt]  [font=\scriptsize]  {$i+2$};
% Text Node
\draw (266,45.4) node [anchor=north west][inner sep=0.75pt]  [font=\scriptsize]  {$n$};
% Text Node
\draw (337,47.4) node [anchor=north west][inner sep=0.75pt]  [font=\scriptsize]  {$1$};
% Text Node
\draw (378,47.4) node [anchor=north west][inner sep=0.75pt]  [font=\scriptsize]  {$i$};
% Text Node
\draw (400,47.4) node [anchor=north west][inner sep=0.75pt]  [font=\scriptsize]  {$i+1$};
% Text Node
\draw (431,47.4) node [anchor=north west][inner sep=0.75pt]  [font=\scriptsize]  {$i+2$};
% Text Node
\draw (477,46.4) node [anchor=north west][inner sep=0.75pt]  [font=\scriptsize]  {$n$};
% Text Node
\draw (296,82.4) node [anchor=north west][inner sep=0.75pt]    {$\Omega _{3}$};
% Text Node
\draw (161,166.4) node [anchor=north west][inner sep=0.75pt]    {$\sigma _{i}^{+} \sigma _{i+1}^{+} \sigma _{i}^{+}$};
% Text Node
\draw (371,166.4) node [anchor=north west][inner sep=0.75pt]    {$\sigma _{i+1}^{+} \sigma _{i}^{+} \sigma _{i+1}^{+}$};


\end{tikzpicture}
    \end{center}
\end{frame}
\begin{frame}
    \begin{center}
        



\tikzset{every picture/.style={line width=0.75pt}} %set default line width to 0.75pt        

\begin{tikzpicture}[x=0.75pt,y=0.75pt,yscale=-1.5,xscale=1.5]
%uncomment if require: \path (0,335); %set diagram left start at 0, and has height of 335

%Straight Lines [id:da6051419366022395] 
\draw    (130,93.33) -- (130,126.67) ;
%Curve Lines [id:da9988810326669739] 
\draw    (230,93.33) .. controls (230,110) and (201,110) .. (200,126.67) ;
%Curve Lines [id:da22799961222723653] 
\draw    (200,93.33) .. controls (201.5,104.33) and (200.5,105.17) .. (212,109.33) ;
%Curve Lines [id:da1490781175691419] 
\draw    (219,111.67) .. controls (224.5,115.17) and (228.5,115.17) .. (230,126.67) ;
%Straight Lines [id:da8207937248086156] 
\draw    (170,93.33) -- (170,126.67) ;
%Straight Lines [id:da048365350972087495] 
\draw    (270,93.33) -- (270,126.67) ;
%Straight Lines [id:da783175719816261] 
\draw    (130,60) -- (130,93.33) ;
%Curve Lines [id:da21948128892783714] 
\draw    (200,60) .. controls (200,76.67) and (171,76.67) .. (170,93.33) ;
%Curve Lines [id:da11000036284050341] 
\draw    (170,60) .. controls (171.5,71) and (170.5,71.83) .. (182,76) ;
%Curve Lines [id:da2016637435430727] 
\draw    (189,78.33) .. controls (194.5,81.83) and (198.5,81.83) .. (200,93.33) ;
%Straight Lines [id:da03184093892635387] 
\draw    (230,60) -- (230,93.33) ;
%Straight Lines [id:da8580736965744533] 
\draw    (270,60) -- (270,93.33) ;
%Straight Lines [id:da3240339948692498] 
\draw    (130,126.67) -- (130,160) ;
%Curve Lines [id:da15803163782768093] 
\draw    (200,126.67) .. controls (200,143.33) and (171,143.33) .. (170,160) ;
%Curve Lines [id:da6791919280524208] 
\draw    (170,126.67) .. controls (171.5,137.67) and (170.5,138.5) .. (182,142.67) ;
%Curve Lines [id:da9636959373597402] 
\draw    (189,145) .. controls (194.5,148.5) and (198.5,148.5) .. (200,160) ;
%Straight Lines [id:da7737137812536472] 
\draw    (230,126.67) -- (230,160) ;
%Straight Lines [id:da7868366212512278] 
\draw    (270,126.67) -- (270,160) ;
%Straight Lines [id:da9117606281933235] 
\draw    (341,60) -- (341,93.33) ;
%Curve Lines [id:da4617900422359481] 
\draw    (441,60) .. controls (441,76.67) and (412,76.67) .. (411,93.33) ;
%Curve Lines [id:da7919077081110307] 
\draw    (411,60) .. controls (412.5,71) and (411.5,71.83) .. (423,76) ;
%Curve Lines [id:da556973599872705] 
\draw    (430,78.33) .. controls (435.5,81.83) and (439.5,81.83) .. (441,93.33) ;
%Straight Lines [id:da7614386709213076] 
\draw    (381,60) -- (381,93.33) ;
%Straight Lines [id:da03345456947894776] 
\draw    (481,60) -- (481,93.33) ;
%Straight Lines [id:da26200967481729853] 
\draw    (341,93.33) -- (341,126.67) ;
%Curve Lines [id:da09553132514781482] 
\draw    (411,93.33) .. controls (411,110) and (382,110) .. (381,126.67) ;
%Curve Lines [id:da2563544636504017] 
\draw    (381,93.33) .. controls (382.5,104.33) and (381.5,105.17) .. (393,109.33) ;
%Curve Lines [id:da13193755972733046] 
\draw    (400,111.67) .. controls (405.5,115.17) and (409.5,115.17) .. (411,126.67) ;
%Straight Lines [id:da9316135114510464] 
\draw    (441,93.33) -- (441,126.67) ;
%Straight Lines [id:da11527130856596324] 
\draw    (481,93.33) -- (481,126.67) ;
%Straight Lines [id:da3239224345124351] 
\draw    (341,126.67) -- (341,160) ;
%Curve Lines [id:da6738308771610565] 
\draw    (441,126.67) .. controls (441,143.33) and (412,143.33) .. (411,160) ;
%Curve Lines [id:da8633246586798422] 
\draw    (411,126.67) .. controls (412.5,137.67) and (411.5,138.5) .. (423,142.67) ;
%Curve Lines [id:da5645812605701747] 
\draw    (430,145) .. controls (435.5,148.5) and (439.5,148.5) .. (441,160) ;
%Straight Lines [id:da31368590787885864] 
\draw    (381,126.67) -- (381,160) ;
%Straight Lines [id:da4310704090732771] 
\draw    (481,126.67) -- (481,160) ;
%Straight Lines [id:da3326166111695229] 
\draw    (290,110) -- (328,110) ;
\draw [shift={(330,110)}, rotate = 180] [color={rgb, 255:red, 0; green, 0; blue, 0 }  ][line width=0.75]    (10.93,-3.29) .. controls (6.95,-1.4) and (3.31,-0.3) .. (0,0) .. controls (3.31,0.3) and (6.95,1.4) .. (10.93,3.29)   ;

% Text Node
\draw (134,101.4) node [anchor=north west][inner sep=0.75pt]    {$\cdots $};
% Text Node
\draw (235,101.4) node [anchor=north west][inner sep=0.75pt]    {$\cdots $};
% Text Node
\draw (134,68.07) node [anchor=north west][inner sep=0.75pt]    {$\cdots $};
% Text Node
\draw (235,68.07) node [anchor=north west][inner sep=0.75pt]    {$\cdots $};
% Text Node
\draw (134,134.73) node [anchor=north west][inner sep=0.75pt]    {$\cdots $};
% Text Node
\draw (235,134.73) node [anchor=north west][inner sep=0.75pt]    {$\cdots $};
% Text Node
\draw (345,68.07) node [anchor=north west][inner sep=0.75pt]    {$\cdots $};
% Text Node
\draw (446,68.07) node [anchor=north west][inner sep=0.75pt]    {$\cdots $};
% Text Node
\draw (345,101.4) node [anchor=north west][inner sep=0.75pt]    {$\cdots $};
% Text Node
\draw (446,101.4) node [anchor=north west][inner sep=0.75pt]    {$\cdots $};
% Text Node
\draw (345,134.73) node [anchor=north west][inner sep=0.75pt]    {$\cdots $};
% Text Node
\draw (446,134.73) node [anchor=north west][inner sep=0.75pt]    {$\cdots $};
% Text Node
\draw (126,46.4) node [anchor=north west][inner sep=0.75pt]  [font=\scriptsize]  {$1$};
% Text Node
\draw (167,46.4) node [anchor=north west][inner sep=0.75pt]  [font=\scriptsize]  {$i$};
% Text Node
\draw (189,46.4) node [anchor=north west][inner sep=0.75pt]  [font=\scriptsize]  {$i+1$};
% Text Node
\draw (220,46.4) node [anchor=north west][inner sep=0.75pt]  [font=\scriptsize]  {$i+2$};
% Text Node
\draw (266,45.4) node [anchor=north west][inner sep=0.75pt]  [font=\scriptsize]  {$n$};
% Text Node
\draw (337,47.4) node [anchor=north west][inner sep=0.75pt]  [font=\scriptsize]  {$1$};
% Text Node
\draw (378,47.4) node [anchor=north west][inner sep=0.75pt]  [font=\scriptsize]  {$i$};
% Text Node
\draw (400,47.4) node [anchor=north west][inner sep=0.75pt]  [font=\scriptsize]  {$i+1$};
% Text Node
\draw (431,47.4) node [anchor=north west][inner sep=0.75pt]  [font=\scriptsize]  {$i+2$};
% Text Node
\draw (477,46.4) node [anchor=north west][inner sep=0.75pt]  [font=\scriptsize]  {$n$};
% Text Node
\draw (296,82.4) node [anchor=north west][inner sep=0.75pt]    {$\Omega _{3}$};
% Text Node
\draw (161,166.4) node [anchor=north west][inner sep=0.75pt]    {$\sigma _{i}^{+} \sigma _{i+1}^{+} \sigma _{i}^{+}$};
% Text Node
\draw (371,166.4) node [anchor=north west][inner sep=0.75pt]    {$\sigma _{i+1}^{+} \sigma _{i}^{+} \sigma _{i+1}^{+}$};


\end{tikzpicture}
    \end{center}
\end{frame}
\begin{frame}{Trenzas Puras}
\begin{block}{Definición}
    Dada la proyección natural vista previamente $\pi:B_n\to S_n$, definimos el \textit{grupo de trenzas puras} como 
    $$P_n:=\ker\pi.$$
\end{block}
\begin{block}{Definición}
    Definimos $A_{i,j}\in P_n$ como la trenza dada por
    $$A_{i,j}:=\sigma_{j-1}\sigma_{j-2}\cdots\sigma_{i+1}\sigma_i^2\sigma_{i+1}^{-1}\cdots\sigma_{j-2}^{-1}\sigma_{j-1}^{-1}.$$ 
    Para $1\leq i<j\leq n$
\end{block}
    
\end{frame}
\begin{frame}
    \begin{center}
        

\tikzset{every picture/.style={line width=0.75pt}} %set default line width to 0.75pt        

\begin{tikzpicture}[x=0.75pt,y=0.75pt,yscale=-1.5,xscale=1.5]
%uncomment if require: \path (0,877); %set diagram left start at 0, and has height of 877

%Straight Lines [id:da43079767492294996] 
\draw    (100,40) -- (100,150) ;
%Straight Lines [id:da8568972494568349] 
\draw    (140,40) -- (140,150) ;
%Straight Lines [id:da39499802412559726] 
\draw    (170,40) -- (170,72) ;
%Straight Lines [id:da06352980068118319] 
\draw    (200,40) -- (200,64) ;
%Straight Lines [id:da5832724793835432] 
\draw    (240,40) -- (240,55) ;
%Straight Lines [id:da9487132303393538] 
\draw    (300,40) -- (300,150) ;
%Straight Lines [id:da2547896303322166] 
\draw    (340,40) -- (340,150) ;
%Curve Lines [id:da9573368491042561] 
\draw    (270,40) .. controls (266.5,75) and (115.5,67) .. (165,101) ;
%Straight Lines [id:da3325229398873272] 
\draw    (170,83) -- (170,150) ;
%Curve Lines [id:da11849363606307461] 
\draw    (174,105) .. controls (223.5,128.5) and (268,133.5) .. (270,150) ;
%Straight Lines [id:da16906643654568176] 
\draw    (200,75) -- (200,110) ;
%Straight Lines [id:da11587080674469141] 
\draw    (200,120) -- (200,150) ;
%Straight Lines [id:da014801284669030856] 
\draw    (240,67) -- (240,124) ;
%Straight Lines [id:da15561552279853996] 
\draw    (240,135) -- (240,150) ;

% Text Node
\draw (105,82.4) node [anchor=north west][inner sep=0.75pt]    {$\cdots $};
% Text Node
\draw (205,82.4) node [anchor=north west][inner sep=0.75pt]    {$\cdots $};
% Text Node
\draw (305,82.4) node [anchor=north west][inner sep=0.75pt]    {$\cdots $};
% Text Node
\draw (97,28.4) node [anchor=north west][inner sep=0.75pt]  [font=\scriptsize]  {$1$};
% Text Node
\draw (129,28.4) node [anchor=north west][inner sep=0.75pt]  [font=\scriptsize]  {$i-1$};
% Text Node
\draw (167,28.4) node [anchor=north west][inner sep=0.75pt]  [font=\scriptsize]  {$i$};
% Text Node
\draw (190,28.4) node [anchor=north west][inner sep=0.75pt]  [font=\scriptsize]  {$i+1$};
% Text Node
\draw (230,28.4) node [anchor=north west][inner sep=0.75pt]  [font=\scriptsize]  {$j-1$};
% Text Node
\draw (271,28.4) node [anchor=north west][inner sep=0.75pt]  [font=\scriptsize]  {$j$};
% Text Node
\draw (289,28.4) node [anchor=north west][inner sep=0.75pt]  [font=\scriptsize]  {$j+1$};
% Text Node
\draw (337,28.4) node [anchor=north west][inner sep=0.75pt]  [font=\scriptsize]  {$n$};


\end{tikzpicture}

    \end{center}
\end{frame}

\begin{frame}{Espacios de Configuración }
\begin{block}{Definición}
    El subespacio de $M^n$ definido como
    $$\mathcal{F}_n(M):=\{(u_1,\ldots,u_n)\in M^n|u_i\neq u_j\text{ para todo }i\neq j\},$$
    se conoce como el \textit{espacio de configuracion} de $n-$tuplas ordenadas de puntos en $M.$
\end{block}
\begin{block}{Definición}
    El grupo fundamental $\pi_1(\mathcal{F}_n(M))$ se conoce como el \textit{grupo de trenzas puras} de $M$ en $n$ cuerdas.
\end{block}
    
\end{frame}
\begin{frame}
    \begin{center}
        

\tikzset{every picture/.style={line width=0.75pt}} %set default line width to 0.75pt        

\begin{tikzpicture}[x=0.75pt,y=0.75pt,yscale=-1.3,xscale=1.3]
%uncomment if require: \path (0,877); %set diagram left start at 0, and has height of 877

%Shape: Axis 2D [id:dp8161459569391534] 
\draw  (150,194) -- (330,194)(168,50) -- (168,210) (323,189) -- (330,194) -- (323,199) (163,57) -- (168,50) -- (173,57)  ;
%Shape: Circle [id:dp3879912667843207] 
\draw  [color={rgb, 255:red, 208; green, 2; blue, 27 }  ,draw opacity=1 ][fill={rgb, 255:red, 208; green, 2; blue, 27 }  ,fill opacity=1 ] (205,194) .. controls (205,191.24) and (207.24,189) .. (210,189) .. controls (212.76,189) and (215,191.24) .. (215,194) .. controls (215,196.76) and (212.76,199) .. (210,199) .. controls (207.24,199) and (205,196.76) .. (205,194) -- cycle ;
%Shape: Circle [id:dp27918612907050666] 
\draw  [color={rgb, 255:red, 74; green, 144; blue, 226 }  ,draw opacity=1 ][fill={rgb, 255:red, 74; green, 144; blue, 226 }  ,fill opacity=1 ] (265,194) .. controls (265,191.24) and (267.24,189) .. (270,189) .. controls (272.76,189) and (275,191.24) .. (275,194) .. controls (275,196.76) and (272.76,199) .. (270,199) .. controls (267.24,199) and (265,196.76) .. (265,194) -- cycle ;
%Shape: Circle [id:dp8020177018586152] 
\draw  [color={rgb, 255:red, 208; green, 2; blue, 27 }  ,draw opacity=1 ][fill={rgb, 255:red, 208; green, 2; blue, 27 }  ,fill opacity=1 ] (250,62) .. controls (250,59.24) and (252.24,57) .. (255,57) .. controls (257.76,57) and (260,59.24) .. (260,62) .. controls (260,64.76) and (257.76,67) .. (255,67) .. controls (252.24,67) and (250,64.76) .. (250,62) -- cycle ;
%Shape: Circle [id:dp7823010661440502] 
\draw  [color={rgb, 255:red, 208; green, 2; blue, 27 }  ,draw opacity=1 ][fill={rgb, 255:red, 208; green, 2; blue, 27 }  ,fill opacity=1 ] (323,62) .. controls (323,59.24) and (325.24,57) .. (328,57) .. controls (330.76,57) and (333,59.24) .. (333,62) .. controls (333,64.76) and (330.76,67) .. (328,67) .. controls (325.24,67) and (323,64.76) .. (323,62) -- cycle ;
%Shape: Circle [id:dp2546903353916321] 
\draw  [color={rgb, 255:red, 74; green, 144; blue, 226 }  ,draw opacity=1 ][fill={rgb, 255:red, 74; green, 144; blue, 226 }  ,fill opacity=1 ] (266,62) .. controls (266,59.24) and (268.24,57) .. (271,57) .. controls (273.76,57) and (276,59.24) .. (276,62) .. controls (276,64.76) and (273.76,67) .. (271,67) .. controls (268.24,67) and (266,64.76) .. (266,62) -- cycle ;
%Shape: Circle [id:dp5671014248101258] 
\draw  [color={rgb, 255:red, 74; green, 144; blue, 226 }  ,draw opacity=1 ][fill={rgb, 255:red, 74; green, 144; blue, 226 }  ,fill opacity=1 ] (307,62) .. controls (307,59.24) and (309.24,57) .. (312,57) .. controls (314.76,57) and (317,59.24) .. (317,62) .. controls (317,64.76) and (314.76,67) .. (312,67) .. controls (309.24,67) and (307,64.76) .. (307,62) -- cycle ;

% Text Node
\draw (191,202.4) node [anchor=north west][inner sep=0.75pt]    {$( 1,0)$};
% Text Node
\draw (251,201.4) node [anchor=north west][inner sep=0.75pt]    {$( 2,0)$};
% Text Node
\draw (136,42.4) node [anchor=north west][inner sep=0.75pt]    {$\mathbb{R}^{2}$};
% Text Node
\draw (244,52.4) node [anchor=north west][inner sep=0.75pt]    {$( \ \ ,\ \ ) \neq ( \ \ ,\ \ )$};


\end{tikzpicture}
    \end{center}
\end{frame} 
\begin{frame}{Equivalencias}
    \begin{block}{Teorema}
    Para $M=\R^2$ tenemos que $\pi_1(\mathcal{F}_n(M))\cong P_n.$
\end{block}
\begin{block}{Definición}
    Dado $\mathcal{F}_n(M)$, definimos
    $$C_n(M):=\mathcal{F}_n(M)/S_n$$
    Donde el cociente es dado por la acción usual de $S_n$ permutando el orden de la entradas. Este espacio se conoce como el espacio de configuración de conjuntos de puntos no ordenados.
\end{block}
\begin{block}{Teorema}
    Para $M=\R^2$ tenemos que
    $$B_n\cong\pi_1(C_n(M),q)$$
    Donde $q$ hace referencia al conjunto de puntos no ordenado $\{(1,0),(2,0),\ldots,(n,0)\}.$
\end{block}

\end{frame}
\begin{frame}
    \begin{center}
        

\tikzset{every picture/.style={line width=0.75pt}} %set default line width to 0.75pt        

\begin{tikzpicture}[x=0.75pt,y=0.75pt,yscale=-1,xscale=1]
%uncomment if require: \path (0,877); %set diagram left start at 0, and has height of 877

%Shape: Rectangle [id:dp23975536104774442] 
\draw   (205,60) -- (360,60) -- (315,90) -- (160,90) -- cycle ;
%Shape: Rectangle [id:dp4742593007559085] 
\draw   (202.5,210) -- (357.5,210) -- (312.5,240) -- (157.5,240) -- cycle ;
%Curve Lines [id:da2338350605887748] 
\draw [line width=1.5]    (190,90) .. controls (189.5,121.5) and (239.5,111) .. (240,140) ;
%Curve Lines [id:da6958858637133809] 
\draw [color={rgb, 255:red, 74; green, 144; blue, 226 }  ,draw opacity=1 ][line width=1.5]    (250,90) .. controls (244.5,105.5) and (241.5,112.5) .. (221,113) ;
%Curve Lines [id:da1801674192268512] 
\draw [color={rgb, 255:red, 74; green, 144; blue, 226 }  ,draw opacity=1 ][line width=1.5]    (206,119) .. controls (192,120.5) and (191,124.5) .. (190,140) ;
%Curve Lines [id:da04494480422128411] 
\draw [color={rgb, 255:red, 74; green, 144; blue, 226 }  ,draw opacity=1 ][line width=1.5]    (190,140) .. controls (190.5,167) and (239.5,152.5) .. (240,170) ;
%Curve Lines [id:da9002028899774802] 
\draw [color={rgb, 255:red, 74; green, 144; blue, 226 }  ,draw opacity=1 ][line width=1.5]    (240,170) .. controls (240.5,197) and (289.5,182.5) .. (290,200) ;
%Curve Lines [id:da7520709988799963] 
\draw [line width=1.5]    (213,155) .. controls (226,152.5) and (234,153.5) .. (240,140) ;
%Curve Lines [id:da5137431769322998] 
\draw [line width=1.5]    (190,240) .. controls (190.5,185) and (190.5,179.5) .. (203,161) ;
%Curve Lines [id:da2225388746369933] 
\draw [color={rgb, 255:red, 208; green, 2; blue, 27 }  ,draw opacity=1 ][line width=1.5]    (250,200) .. controls (250.5,226) and (299.5,217.5) .. (300,240) ;
%Curve Lines [id:da19326513425460334] 
\draw [color={rgb, 255:red, 74; green, 144; blue, 226 }  ,draw opacity=1 ][line width=1.5]    (240,240) .. controls (243,223.5) and (248.5,221) .. (260,220) ;
%Curve Lines [id:da40134196403228717] 
\draw [color={rgb, 255:red, 74; green, 144; blue, 226 }  ,draw opacity=1 ][line width=1.5]    (290,200) .. controls (290,210.5) and (287.5,217.5) .. (269,216) ;
%Curve Lines [id:da665280040996851] 
\draw [color={rgb, 255:red, 208; green, 2; blue, 27 }  ,draw opacity=1 ][line width=1.5]    (250,200) .. controls (250,193.5) and (256,190) .. (262,190) ;
%Curve Lines [id:da46296116559591394] 
\draw [color={rgb, 255:red, 208; green, 2; blue, 27 }  ,draw opacity=1 ][line width=1.5]    (272,185) .. controls (297.5,175) and (298.5,127.5) .. (300,90) ;
%Straight Lines [id:da17288821057020853] 
\draw [line width=1.5]    (370,110) -- (390,110) ;
%Straight Lines [id:da7810709935908421] 
\draw [color={rgb, 255:red, 74; green, 144; blue, 226 }  ,draw opacity=1 ][line width=1.5]    (370,140) -- (390,140) ;
%Straight Lines [id:da4526710696777342] 
\draw [color={rgb, 255:red, 208; green, 2; blue, 27 }  ,draw opacity=1 ][line width=1.5]    (370,170) -- (390,170) ;

% Text Node
\draw (371,70.4) node [anchor=north west][inner sep=0.75pt]    {$\mathbb{R}^{2}$};
% Text Node
\draw (361,220.4) node [anchor=north west][inner sep=0.75pt]    {$\mathbb{R}^{2}$};
% Text Node
\draw (392,99.4) node [anchor=north west][inner sep=0.75pt]    {$\alpha _{1}( t)$};
% Text Node
\draw (392,129.4) node [anchor=north west][inner sep=0.75pt]    {$\alpha _{2}( t)$};
% Text Node
\draw (392,159.4) node [anchor=north west][inner sep=0.75pt]    {$\alpha _{3}( t)$};


\end{tikzpicture}

    \end{center}
\end{frame}
\begin{frame}
    \begin{center}
        

\tikzset{every picture/.style={line width=0.75pt}} %set default line width to 0.75pt        

\begin{tikzpicture}[x=0.75pt,y=0.75pt,yscale=-1,xscale=1]
%uncomment if require: \path (0,877); %set diagram left start at 0, and has height of 877

%Shape: Rectangle [id:dp23975536104774442] 
\draw   (205,60) -- (360,60) -- (315,90) -- (160,90) -- cycle ;
%Shape: Rectangle [id:dp4742593007559085] 
\draw   (202.5,210) -- (357.5,210) -- (312.5,240) -- (157.5,240) -- cycle ;
%Straight Lines [id:da17288821057020853] 
\draw [line width=1.5]    (370,110) -- (390,110) ;
%Straight Lines [id:da7810709935908421] 
\draw [color={rgb, 255:red, 74; green, 144; blue, 226 }  ,draw opacity=1 ][line width=1.5]    (370,140) -- (390,140) ;
%Curve Lines [id:da18292473776224472] 
\draw [color={rgb, 255:red, 74; green, 144; blue, 226 }  ,draw opacity=1 ][line width=1.5]    (280,90) .. controls (279.5,168) and (201,151) .. (200,240) ;
%Curve Lines [id:da8202799075901603] 
\draw [line width=1.5]    (200,90) .. controls (201.5,130) and (206,149.5) .. (232,160) ;
%Curve Lines [id:da7930766509930773] 
\draw [line width=1.5]    (244,168) .. controls (266.5,176.5) and (277.5,194) .. (280,240) ;

% Text Node
\draw (371,70.4) node [anchor=north west][inner sep=0.75pt]    {$\mathbb{R}^{2}$};
% Text Node
\draw (361,220.4) node [anchor=north west][inner sep=0.75pt]    {$\mathbb{R}^{2}$};
% Text Node
\draw (392,99.4) node [anchor=north west][inner sep=0.75pt]    {$u_{1}( t)$};
% Text Node
\draw (392,129.4) node [anchor=north west][inner sep=0.75pt]    {$u_{2}( t)$};


\end{tikzpicture}

    \end{center}
\end{frame}

\begin{frame}{Automorfismos de $F_n$}

 \begin{block}{Definición}
    Decimos que un automorfismo $\phi:F_n\to F_n$ es un \textit{automorfismo de trenzas} si satisface las siguientes condiciones
    \begin{itemize}
        \item[i)] Existe $\mu\in S_n$ tal que $\phi(a_k)$ es conjugado en $F_n$ a $a_{\mu(k)}$ para todo $k\in{1,2,\ldots,n}$
        \item[ii)]$\phi(a_1\dots a_n)=a_1\dots a_n.$
    \end{itemize}
\end{block}  
\begin{block}{Proposición}
       $\tilde B_n$ es un grupo con la composición. 
    \end{block}
\end{frame}

\begin{frame}{Mapping Class Group}
\begin{block}{Definición}
   Un \textit{auto-homeomorfismo} de la pareja $(M,Q)$ es un homeomorfismo $f:M\to M$ tal que
    \begin{itemize}
        \item[i)] Para todo $x\in\partial M$,\, $f(x)=x$.
        \item[ii)] $f(Q)=Q.$ 
        \item[iii)] Preserva la orientacion.
    \end{itemize}
\end{block} 
\begin{block}{Definición}
    El \textit{Mapping Class Group} $\mathcal{M}(M,Q)$ es el conjunto de clases de isotopia de automorfismos con la composicion de funciuones como operacion.
\end{block}
\end{frame}
\begin{frame}
\begin{block}{Definición}
    Decimos que $\alpha$ es un \textit{arco generador} en $(M,Q)$ si
    \begin{itemize}
        \item $\alpha\subset M$ con $\alpha$ homeomorfico a $I.$
        \item $\alpha\cap(Q\cup\partial M)=\{x_1,x_2\}\subset Q$
    \end{itemize}
\end{block}
\begin{block}{Definición}
    Dado un arco generador $\alpha$ definimos el \textit{medio-giro} como
    $$\tau_\alpha\colon(M,Q)\to(M,Q)$$
    Tal que dada una vecindad pequeña $U$ de $\alpha$, que identificamos de manera homeomorfa con $\{z\in\C\colon|z|<1\}$ tal que $\alpha=\left[-\dfrac{1}{2},\dfrac{1}{2}\right]$ y la orientacion de $M$ sea en contra de las manecillas del reloj, tal que fuera de $U$, $\tau_\alpha$ es la identidad, Si $|z|\leq\dfrac{1}{2}$ es enviado a $-z$, y para $\dfrac{1}{2}\leq|z|<1$ es enviado a $ze^{-2\pi i|z|}$
\end{block}
  
   \end{frame}  
    \begin{frame}
    \begin{center}
        

\tikzset{every picture/.style={line width=0.75pt}} %set default line width to 0.75pt        

\begin{tikzpicture}[x=0.75pt,y=0.75pt,yscale=-1,xscale=1]
%uncomment if require: \path (0,877); %set diagram left start at 0, and has height of 877

%Shape: Circle [id:dp9497408554436547] 
\draw   (120,155) .. controls (120,108.06) and (158.06,70) .. (205,70) .. controls (251.94,70) and (290,108.06) .. (290,155) .. controls (290,201.94) and (251.94,240) .. (205,240) .. controls (158.06,240) and (120,201.94) .. (120,155) -- cycle ;
%Straight Lines [id:da6924922424889027] 
\draw [color={rgb, 255:red, 74; green, 144; blue, 226 }  ,draw opacity=1 ]   (200,260) -- (200,52) ;
\draw [shift={(200,50)}, rotate = 90] [color={rgb, 255:red, 74; green, 144; blue, 226 }  ,draw opacity=1 ][line width=0.75]    (10.93,-3.29) .. controls (6.95,-1.4) and (3.31,-0.3) .. (0,0) .. controls (3.31,0.3) and (6.95,1.4) .. (10.93,3.29)   ;
%Straight Lines [id:da4879924609046534] 
\draw    (150,160) -- (258,160) ;
\draw [shift={(260,160)}, rotate = 180] [color={rgb, 255:red, 0; green, 0; blue, 0 }  ][line width=0.75]    (10.93,-3.29) .. controls (6.95,-1.4) and (3.31,-0.3) .. (0,0) .. controls (3.31,0.3) and (6.95,1.4) .. (10.93,3.29)   ;
%Straight Lines [id:da7522321699548209] 
\draw    (310,160) -- (348,160) ;
\draw [shift={(350,160)}, rotate = 180] [color={rgb, 255:red, 0; green, 0; blue, 0 }  ][line width=0.75]    (10.93,-3.29) .. controls (6.95,-1.4) and (3.31,-0.3) .. (0,0) .. controls (3.31,0.3) and (6.95,1.4) .. (10.93,3.29)   ;
%Shape: Circle [id:dp5441253380968003] 
\draw   (360,155) .. controls (360,108.06) and (398.06,70) .. (445,70) .. controls (491.94,70) and (530,108.06) .. (530,155) .. controls (530,201.94) and (491.94,240) .. (445,240) .. controls (398.06,240) and (360,201.94) .. (360,155) -- cycle ;
%Straight Lines [id:da9790573072695838] 
\draw    (500,160) -- (392,160) ;
\draw [shift={(390,160)}, rotate = 360] [color={rgb, 255:red, 0; green, 0; blue, 0 }  ][line width=0.75]    (10.93,-3.29) .. controls (6.95,-1.4) and (3.31,-0.3) .. (0,0) .. controls (3.31,0.3) and (6.95,1.4) .. (10.93,3.29)   ;
%Straight Lines [id:da286317178551147] 
\draw [color={rgb, 255:red, 74; green, 144; blue, 226 }  ,draw opacity=1 ]   (440,105) -- (440,215) ;
%Curve Lines [id:da6408676567838075] 
\draw [color={rgb, 255:red, 74; green, 144; blue, 226 }  ,draw opacity=1 ]   (440,215) .. controls (284.57,155.1) and (436.4,99.62) .. (439.94,51.46) ;
\draw [shift={(440,50)}, rotate = 90.59] [color={rgb, 255:red, 74; green, 144; blue, 226 }  ,draw opacity=1 ][line width=0.75]    (10.93,-3.29) .. controls (6.95,-1.4) and (3.31,-0.3) .. (0,0) .. controls (3.31,0.3) and (6.95,1.4) .. (10.93,3.29)   ;
%Curve Lines [id:da7810815567546844] 
\draw [color={rgb, 255:red, 74; green, 144; blue, 226 }  ,draw opacity=1 ]   (440,105) .. controls (621.5,165) and (424.5,199) .. (440,260) ;

% Text Node
\draw (111,72.4) node [anchor=north west][inner sep=0.75pt]    {$D^{2}$};
% Text Node
\draw (318,138.4) node [anchor=north west][inner sep=0.75pt]    {$\tau _{\alpha }$};
% Text Node
\draw (351,72.4) node [anchor=north west][inner sep=0.75pt]    {$D^{2}$};


\end{tikzpicture}
    \end{center}
\end{frame}
   
   \begin{frame}{Equivalencia MCG y Automorfismos}
   \begin{block}{Teorema}
    Para $n\geq 1$, los homomorfismos $\eta,\rho$ son isomorfismos que hacen que el siguiente diagrama conmute
     \begin{center}
         

\tikzset{every picture/.style={line width=0.75pt}} %set default line width to 0.75pt        

\begin{tikzpicture}[x=0.75pt,y=0.75pt,yscale=-1,xscale=1]
%uncomment if require: \path (0,877); %set diagram left start at 0, and has height of 877

%Straight Lines [id:da3495190020906457] 
\draw    (190,130) -- (190,188) ;
\draw [shift={(190,190)}, rotate = 270] [color={rgb, 255:red, 0; green, 0; blue, 0 }  ][line width=0.75]    (10.93,-3.29) .. controls (6.95,-1.4) and (3.31,-0.3) .. (0,0) .. controls (3.31,0.3) and (6.95,1.4) .. (10.93,3.29)   ;
%Straight Lines [id:da11668079361738581] 
\draw    (210,120) -- (308.36,188.85) ;
\draw [shift={(310,190)}, rotate = 214.99] [color={rgb, 255:red, 0; green, 0; blue, 0 }  ][line width=0.75]    (10.93,-3.29) .. controls (6.95,-1.4) and (3.31,-0.3) .. (0,0) .. controls (3.31,0.3) and (6.95,1.4) .. (10.93,3.29)   ;
%Straight Lines [id:da13478561325303762] 
\draw    (240,210) -- (308,210) ;
\draw [shift={(310,210)}, rotate = 180] [color={rgb, 255:red, 0; green, 0; blue, 0 }  ][line width=0.75]    (10.93,-3.29) .. controls (6.95,-1.4) and (3.31,-0.3) .. (0,0) .. controls (3.31,0.3) and (6.95,1.4) .. (10.93,3.29)   ;

% Text Node
\draw (178,102.4) node [anchor=north west][inner sep=0.75pt]  [font=\Large]  {$B_{n}$};
% Text Node
\draw (151,195.4) node [anchor=north west][inner sep=0.75pt]  [font=\large]  {$\mathcal{M}( D,Q_{n})$};
% Text Node
\draw (320,182.4) node [anchor=north west][inner sep=0.75pt]  [font=\Large]  {$\tilde{B}_{n}$};
% Text Node
\draw (261,212.4) node [anchor=north west][inner sep=0.75pt]    {$\rho $};
% Text Node
\draw (173,145.4) node [anchor=north west][inner sep=0.75pt]    {$\eta $};


\end{tikzpicture}
     \end{center}
\end{block}
       
    \end{frame}
    \begin{frame}
    \begin{center}
        

\tikzset{every picture/.style={line width=0.75pt}} %set default line width to 0.75pt        

\begin{tikzpicture}[x=0.75pt,y=0.75pt,yscale=-1,xscale=1]
%uncomment if require: \path (0,877); %set diagram left start at 0, and has height of 877

%Shape: Circle [id:dp37495404303702484] 
\draw   (110,145) .. controls (110,75.96) and (165.96,20) .. (235,20) .. controls (304.04,20) and (360,75.96) .. (360,145) .. controls (360,214.04) and (304.04,270) .. (235,270) .. controls (165.96,270) and (110,214.04) .. (110,145) -- cycle ;
%Straight Lines [id:da09124629345358082] 
\draw    (130,150) -- (170,150) ;
%Straight Lines [id:da35678252994237136] 
\draw    (170,150) -- (210,150) ;
%Straight Lines [id:da5587533975361445] 
\draw    (250,152) -- (290,152) ;
%Straight Lines [id:da14803740874356064] 
\draw    (290,152) -- (330,152) ;

% Text Node
\draw (142,130.4) node [anchor=north west][inner sep=0.75pt]    {$\alpha _{1}$};
% Text Node
\draw (181,130.4) node [anchor=north west][inner sep=0.75pt]    {$\alpha _{2}$};
% Text Node
\draw (221,136.4) node [anchor=north west][inner sep=0.75pt]    {$\dotsc $};
% Text Node
\draw (262,132.4) node [anchor=north west][inner sep=0.75pt]    {$\alpha _{n-2}$};
% Text Node
\draw (301,132.4) node [anchor=north west][inner sep=0.75pt]    {$\alpha _{n-1}$};
% Text Node
\draw (118,154.4) node [anchor=north west][inner sep=0.75pt]  [font=\tiny]  {$( 1,0)$};
% Text Node
\draw (159,155.4) node [anchor=north west][inner sep=0.75pt]  [font=\tiny]  {$( 2,0)$};
% Text Node
\draw (280,158.4) node [anchor=north west][inner sep=0.75pt]  [font=\tiny]  {$( n-1,0)$};
% Text Node
\draw (321,157.4) node [anchor=north west][inner sep=0.75pt]  [font=\tiny]  {$( n,0)$};


\end{tikzpicture}

    \end{center}
\end{frame} 
    

\begin{frame}{$B_3$ y $PSL(2,\Z)$}
    Primero recordemos por la presentacion de Artin que
    $$B_3=\langle \sigma_1,\sigma_2\,|\,\sigma_1\sigma_2\sigma_1=\sigma_2\sigma_1\sigma_2\rangle$$
    Recordemos que esta relación resume el movimiento $\Omega_3$. Primero veamos alguna presentacion mas conveniente. Definamos
    $$x=\sigma_1\sigma_2\sigma_1\hspace{5mm}y=\sigma_1\sigma_2$$
    Note que $x=y\sigma_1$, luego $\sigma_1=y^{-1}x,$ mientras que $$\sigma_2=\sigma_1^{-1}x\sigma_1^{-1}=(x^{-1}y)x(x^{-1}y)=x^{-1}y^2$$
    Luego como ambos generadores los podemos reescribir en términos de $x$ y $y$ tenemos la siguiente presentación equivalente
    $$B_3=\langle x,y\,|\,x^2=y^3\rangle$$
\end{frame}
\begin{frame}{}
    \begin{block}{Proposición}
         $$Z(B_3)=\langle y^3 \rangle$$
    \end{block}
    \begin{block}{Teorema}
         Dados $B_3$ y $SL(2,\Z)$ existe un unico homomorfismo, que desciende en un isomorfismo
        $$B_3/Z(B_3)\cong PSL(2,\Z).$$
    \end{block}
\end{frame}
 \begin{frame}
    \begin{center}
        

\tikzset{every picture/.style={line width=0.75pt}} %set default line width to 0.75pt        

\begin{tikzpicture}[x=0.75pt,y=0.75pt,yscale=-1.7,xscale=1.7]
%uncomment if require: \path (0,877); %set diagram left start at 0, and has height of 877

%Curve Lines [id:da12746258618174844] 
\draw    (220,80) .. controls (221.5,129.5) and (140,129) .. (140,170) ;
%Curve Lines [id:da7267308029118377] 
\draw    (180,80) .. controls (180,118.5) and (117.5,124) .. (148,143) ;
%Curve Lines [id:da11813265743405121] 
\draw    (180,170) .. controls (178.5,158) and (170,152) .. (160,150) ;
%Curve Lines [id:da6401519497181539] 
\draw    (140,80) .. controls (143.5,94.5) and (148.5,103.5) .. (160,107) ;
%Curve Lines [id:da3785360469498672] 
\draw    (190,128) .. controls (205,136.5) and (213.5,147) .. (220,170) ;
%Straight Lines [id:da5224494229876718] 
\draw    (168,113) -- (182,123) ;
%Straight Lines [id:da422154150629262] 
\draw    (120,80) -- (240,80) ;
%Straight Lines [id:da5681564586047843] 
\draw    (120,170) -- (240,170) ;

% Text Node
\draw (168,180.4) node [anchor=north west][inner sep=0.75pt]    {$\Delta _{3}$};


\end{tikzpicture}
    \end{center}
\end{frame}
\begin{frame}{Hechos del Grupo de Trenzas}

\begin{block}{Corolario}
     $P_n$ es generado por $A_{i,j}$ para $1\leq i<j\leq n$    
    \end{block}
    \begin{block}{Teorema}
     Si $n\geq 3$ $Z(B_n)=Z(P_n)=\langle \Delta_n^2\rangle$, donde
    $$\Delta_n=(\sigma_1\sigma_2\cdots\sigma_{n-1})\cdots(\sigma_1\sigma_2)\sigma_1$$   
    \end{block}
    \begin{block}{Corolario}
      Para $m\neq n$, $B_m$ no es isomorfo a $B_n.$    
    \end{block}
    \begin{block}{Teorema}
     $B_n$ es un grupo libre de torsión.  
    \end{block}
    
\end{frame}
\begin{frame}{El espacio $C_n(\R^2)$}
 Finalizamos con una breve descripción del espacio de configuración por medio de polinomios, que nos da una relación muy interesante hacia la geometría algebraica. Si hacemos la identificación natural de $\R^2=\C$, consideremos el siguiente polinomio simétrico
 $$p_k(u)=(-1)^k\sum_{1\leq i_1<\cdots<i_k\leq n}u_{i_1}\cdots u_{i_k}.$$
 Donde $u\in \mathcal{F}_n(\C)$ y $k=1,\ldots n$, note que por ser simétrico estas funciones $p_k$ son invariantes bajo la acción de $S_n$, por lo que inducen una función $C_n(\C)\to \C^n $. Resulta que esta función es un homeomorfismo al conjunto de los polinomios monoicos con raíces diferentes de grano $n$ y coeficientes complejos. Siendo así $B_n$ el grupo fundamental de un conjunto clásico en la geometría algebraica.
\end{frame}



\begin{frame}{Referencias}
\begin{thebibliography}{99}

\bibitem[Kassel y Turaev(2008)]{KasselTuraev2008}
Kassel, C. y Turaev, V.
\newblock \emph{Braid Groups}.
\newblock Graduate Texts in Mathematics, vol. 247, Springer, 2008.
\newblock doi:10.1007/978-0-387-68548-9.

\bibitem[Birman y Brendle(2004)]{BirmanBrendle2004}
Birman, J.~S. y Brendle, T.~E.
\newblock Braids: A Survey.
\newblock \emph{arXiv Mathematics e-prints}, 2004.
\newblock \url{https://arxiv.org/abs/math/0409205}.

\bibitem[González-Meneses(2010)]{gonzalezmeneses2010basicresultsbraidgroups}
González-Meneses, J.
\newblock Basic results on braid groups.
\newblock \emph{arXiv:1010.0321} [math.GT].
\newblock Disponible en: \url{https://arxiv.org/abs/1010.0321}.

\bibitem[Artin(1925)]{Artin1925}
Artin, E.
\newblock Theorie der Zöpfe.
\newblock \emph{Abh. Math. Sem. Univ. Hamburg} 4 (1925), 47--72.

\bibitem[Artin(1947a)]{Artin1947a}
Artin, E.
\newblock Theory of braids.
\newblock \emph{Ann. of Math.} 48 (1947), 101--126.

\bibitem[Artin(1947b)]{Artin1947b}
Artin, E.
\newblock Braids and permutations.
\newblock \emph{Ann. of Math.} 48 (1947), 643--649.

\end{thebibliography}

\end{frame}

\end{document}




