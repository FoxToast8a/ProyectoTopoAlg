\documentclass[xcolor=dvipsnames,10pt]{beamer}
\usepackage[utf8]{inputenc}
\usepackage[spanish]{babel}
\usepackage{multirow,rotating}
\usepackage{amsfonts,amsmath,amssymb,amsthm}
\usepackage{color}
\usepackage{hyperref}
\usepackage{fourier,libertine}
\usepackage{transparent}
\usepackage{tikz-cd}
\usepackage{array}
\usepackage{algpseudocode}
\usepackage{siunitx}
\usepackage{algorithm}
\usepackage{mathtools,nccmath}%
\usetheme{Madrid}
\usepackage{xcolor}
\usepackage{natbib}
\usepackage{hyperref}
\usepackage{ragged2e}
\usefonttheme{serif}
\usefonttheme{professionalfonts}
\newcommand\T{\ensuremath{\mathbb{T}}}
\newcommand\N{\ensuremath{\mathbb{N}}}
\newcommand\R{\ensuremath{\mathbb{R}}}
\newcommand\Z{\ensuremath{\mathbb{Z}}}
\renewcommand\O{\ensuremath{\emptyset}}
\newcommand\Q{\ensuremath{\mathbb{Q}}}
\newcommand\C{\ensuremath{\mathbb{C}}}
\newcommand\Hs{\ensuremath{\mathbb{H}}}
% set color ------------------------------------------------------------------
\definecolor{DarkBlue}{rgb}{0.04706, 0.13725, 0.26667} 
\definecolor{cadmiumred}{rgb}{0.45 , 0.12, 0.23}
\definecolor{armygreen}{rgb}{0.10, 0.27, 0.19}
\definecolor{coolblack}{rgb}{0.0, 0.18, 0.39}
\definecolor{lilac}{rgb}{0.33, 0.12, 0.36}
\definecolor{negro}{rgb}{0, 0, 0}
\usecolortheme[named=coolblack]{structure}
\setbeamercolor{block title}{bg=coolblack!100!white,fg=white}
\setbeamercolor{block body}{bg=coolblack!11!white}

%----------------------------------------------------------------------------
\setbeamerfont{title}{size=\large}
\setbeamerfont{subtitle}{size=\small}
\setbeamerfont{author}{size=\small}
\setbeamerfont{date}{size=\footnotesize}
\setbeamerfont{institute}{size=\footnotesize}
\title[Universidad Nacional de Colombia]{Grupos de Trenzas y Espacios de Configuración}
\subtitle{Un primer acercamiento a las nociones algebraicas, geométricas y topológicas}
\author[Topología Algebraica]{
Edgar Santiago Ochoa Quiroga}

\date[\textcolor{white}{Diciembre/2025}]


\begin{document}
\maketitle
\begin{frame}{Introducción}
Cuando hablamos de Trenzas en un ámbito general es completamente natural que la primera imagen que se pase por la cabeza es la de una trenza de cabello. La mas clásica se hace agarrando tres mechones, donde los mechones externos pasan hacia el centro intercalándose.  
\begin{figure}
    \centering
    \includegraphics[width=0.5\linewidth]{trenzado_tres.jpg}
    
\end{figure}
La primera aparición con mención propia de los grupos de trenzas se las debemos a Emil Artin, quien en 1925 los introdujo para modelar como se entrelazaban múltiples cuerdas en un espacio euclidiano 3-dimensional, estas cuerdas es lo que conocemos como trenzas.
\end{frame}

\begin{frame}{Grupos de Trenzas geometricos $\mathcal{B}_n$}
\begin{block}{Definición}
    Una \textbf{trenza geometrica} con $n\geq 1$ cuerdas es un conjunto $b\subset \R^2\times I$ formado por $n$ intervalos topológicos disyuntos llamados las \textit{cuerdas} de $b$, tales que la proyeccion $\R^2\times I\to I$, envia cada cuerda de manera homeomorfica a $I$, y ademas
     \begin{align*}
         b\cap(\R^2\times \{0\})=\{(i,0,0)|i=1,2,\ldots n\},\\
         b\cap(\R^2\times \{0\})=\{(i,0,1)|i=1,2,\ldots n\}.
     \end{align*}
\end{block}
  
\end{frame}

\begin{frame}{Equivalencia de Trenzas Geometricas}
    \begin{block}{Definición}
        Dos trenzas $b$ y $b^\prime$ son isotopicas si existe una función continua $F:b\times I\to R^2\times I$, tal que para cada $s\in I$, la función $F_s:=F(-,s)$ es un embedimiento el cual su imagen es una trenza geométrica de $n$ cuerdas, $F_0=Id_b$ y $F_1(b)=b^\prime.$\\
    Tanto $F$ como la familia de trenzas geometricas $\{F_s(b)\}_{s\in I}$ se conocen como una isotopia de $b$ a $b^\prime$.
    \end{block}
    \begin{block}{Proposición}
        La isotopia entre trenzas geométricas define una relación de equivalencia.
    \end{block}
\end{frame}
\begin{frame}{Estructura de Monoide}

\begin{block}{Definicion}
    Dadas 2 trenzas geométricas con $n$ cuerdas $b_1,b_2$, definimos su producto como el conjunto de puntos $(x,y,t)\in \R^2\times I$ tales que si $t\in[0,1/2]$ entonces $(x,y,2t)\in b_1$, y en caso donde $t\in[1/2,1]$ tenemos que $(x,y,2t-1)\in b_2$.
\end{block}
\begin{block}{Proposición}
    Dadas $b_1,b_2,b_1^\prime,b_2^\prime$, donde $b_i$ es isotopica a $b_i^\prime$ para $i=1,2$, entonces $b_1b_2$ es isotopica a $b_1^\prime b^\prime_2$. 
\end{block}
\end{frame}
\begin{frame}{Diagramas de Trenzas}
\begin{block}{Definición}
    Un \textbf{diagrama de trenzas} de $n$ cuerdas es un conjunto $D\subset \R\times I$, vista como la union de $n$ intervalos topologicos llamados las cuerdas de $D$ donde se cumplen las siguientes condiciones.
    \begin{itemize}
        \item La proyeccion $\R\times I\to I$ envia cada cuerda homeomorficamente a $I.$
        \item Cada punto $\{1,2,\ldots\}\times\{0,1\}$ es el extremo de una unica cuerda.
        \item Todo punto de $\R\times I$ pertenece a lo maximo a 2 cuerdas. En cada punto de interseccion las cuerdas se cruzan de manera transversal y una de ellas se difumina para indicar que pasa por debajo, mientras la que se ve continua indica que pasa por arriba. 
        \end{itemize} 
\end{block}
\end{frame}
\begin{frame}{}
    \begin{block}{Definicion}
        Dos diagramas de trenzas $D$ y $D^\prime$ se dicen isotopicos  si existe una funcion continua $F:D\times I\to R\times I$ tal que para cada $s\in I$, $D_s=F(D\times\{s\})\subset \R\times I$ es un diagrama de trenzas con la misma cantidad cuerdas, ademas $D_0=D$ y $D_1=D^\prime.$ 
    \end{block}

    \begin{block}{Definición}
        Definimos $\Omega_2$ un movimiento donde tomamos dos cuerdas del diagrama y creamos dos cruces nuevos transversales, pasando una de las cuerdas del diagrama por debajo de la otra, mientras que a $\Omega_3$ es un movimiento que involucra a tres cuerdas y preserva el numero de cruces transversales pero invierte el diagrama de manera reflexiva.  
    \end{block}
\end{frame}
\begin{frame}{Diagramas y Trenzas Geometricas}
    \begin{block}{Definicion}
        Dados dos diagramas $D$ y $D^\prime$ decimos que son $R$-equivalentes si $D$ se puede transformas por medio de una secuencia finita de isotopias y movimientos de Reidemeister a $D^\prime$
    \end{block}
    \begin{block}{Teorema}
      Dos diagramas de trenzas representan trenzas geométricas isotopicas si y solo si estos diagramas son $R$-equivalentes.  
    \end{block}
\end{frame}
\begin{frame}{}
    \begin{block}{Teorema}
        $\mathcal{B}_n$ es un grupo.                                                            
    \end{block}
\end{frame}
\begin{frame}{Grupo de Trenzas de Artin $B_n$}

\begin{block}{Definición}
    El grupo de trenzas de Artin $B_n$ es el grupo generado por $n-1$ generadores $\sigma_i$ con $i=1,2,\ldots n-1$ y las relaciones
    $$\sigma_i\sigma_j=\sigma_j\sigma_i,$$
    para todo $i,j=1,2,\ldots n-1$ con $|i-j|\geq 2,$ y
    $$\sigma_i\sigma_{i+1}\sigma_i=\sigma_{i+1}\sigma_i\sigma_{i+1},$$
    para $i=1,2,\ldots n-2.$
\end{block}
\end{frame}
\begin{frame}{}
    \begin{block}{Lema}
        Si existen elementos $\{s_i|i=1,\ldots,n-1\}$ en un grupo $G$ tales que se satisfacen las relaciones de trenzas, entonces existe un unico homomorfismo de grupos $f:B_n\to G$, tal que $s_i=f(\sigma_i).$
    \end{block}
    \begin{block}{Lema}
         Las trasposiciones simples $s_i=(i\,i+1)\in S_n$, generan $S_n$ y satisfacen las relaciones de trenzas.
    \end{block}
    \begin{block}{Teorema}
        El grupo $B_n$ para $n\geq 3$ no es abeliano
    \end{block}
\end{frame}
\begin{frame}{Equivalencia entre $\mathcal{B}_n$ y $B_n$}
    \begin{block}{Proposicion}
        Los elementos $\sigma_1^+,\ldots,\sigma_{n-1}^+\in \mathcal{B}_n$ satisfacen las relaciones de trenzas. 
    \end{block}
    \begin{block}{Teorema}
        Para $\varepsilon=\pm,$ existe un unico homomorfismo $\varphi_{\varepsilon}:B_n\to\mathcal{B}_n$ tal que \\$\varphi_{\varepsilon}(\sigma_i)=\sigma_i^{\varepsilon}$ para todo $i=1,\ldots n-1.$ Ademas el homorfismo $\varphi_\varepsilon$ resulta ser un isormorfismo. 
    \end{block}
\end{frame}
\begin{frame}{Trenzas Puras $P_n$}
\begin{block}{Definición}
    Dada la proyección natural vista previamente $\pi:B_n\to S_n$, definimos el \textit{grupo de trenzas puras} como 
    $$P_n:=\ker\pi.$$
\end{block}
    
\end{frame}


    

\begin{frame}{$B_3$ y $PSL(2,\Z)$}
    Primero recordemos por la presentacion de Artin que
    $$B_3=\langle \sigma_1,\sigma_2\,|\,\sigma_1\sigma_2\sigma_1=\sigma_2\sigma_1\sigma_2\rangle$$
    Recordemos que esta relación resume el movimiento $\Omega_3$. Primero veamos alguna presentacion mas conveniente. Definamos
    $$x=\sigma_1\sigma_2\sigma_1\hspace{5mm}y=\sigma_1\sigma_2$$
    Note que $x=y\sigma_1$, luego $\sigma_1=y^{-1}x,$ mientras que $$\sigma_2=\sigma_1^{-1}x\sigma_1^{-1}=(x^{-1}y)x(x^{-1}y)=x^{-1}y^2$$
    Luego como ambos generadores los podemos reescribir en términos de $x$ y $y$ tenemos la siguiente presentación equivalente
    $$B_3=\langle x,y\,|\,x^2=y^3\rangle$$
\end{frame}
\begin{frame}{}
    \begin{block}{Proposición}
         $$Z(B_3)=\langle y^3 \rangle$$
    \end{block}
\end{frame}



\begin{frame}{Referencias}
\begin{thebibliography}{99}

\bibitem[Kassel y Turaev(2008)]{KasselTuraev2008}
Kassel, C. y Turaev, V.
\newblock \emph{Braid Groups}.
\newblock Graduate Texts in Mathematics, vol. 247, Springer, 2008.
\newblock doi:10.1007/978-0-387-68548-9.

\bibitem[Birman y Brendle(2004)]{BirmanBrendle2004}
Birman, J.~S. y Brendle, T.~E.
\newblock Braids: A Survey.
\newblock \emph{arXiv Mathematics e-prints}, 2004.
\newblock \url{https://arxiv.org/abs/math/0409205}.

\bibitem[González-Meneses(2010)]{gonzalezmeneses2010basicresultsbraidgroups}
González-Meneses, J.
\newblock Basic results on braid groups.
\newblock \emph{arXiv:1010.0321} [math.GT].
\newblock Disponible en: \url{https://arxiv.org/abs/1010.0321}.

\bibitem[Artin(1925)]{Artin1925}
Artin, E.
\newblock Theorie der Zöpfe.
\newblock \emph{Abh. Math. Sem. Univ. Hamburg} 4 (1925), 47--72.

\bibitem[Artin(1947a)]{Artin1947a}
Artin, E.
\newblock Theory of braids.
\newblock \emph{Ann. of Math.} 48 (1947), 101--126.

\bibitem[Artin(1947b)]{Artin1947b}
Artin, E.
\newblock Braids and permutations.
\newblock \emph{Ann. of Math.} 48 (1947), 643--649.

\end{thebibliography}

\end{frame}

\end{document}




